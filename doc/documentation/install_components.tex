\chapter{Components used by OpenCms}
\label{components}

This chapter describes all external parts required to run OpenCms.
If you want to prepare an installation, please collect all
components listed here. If you download a major release of OpenCms
(e.g. version 4.4), Xerces, JavaMail and FOP will be shipped with
the distribution.

\section{Operating System}
OpenCms is implemented in 100\% Java. Since Java should be platform independent, OpenCms should run
on all operating systems supporting Java. Currently, we have installations running on
Red Hat Linux 6.1 (\rqhttp{http://www.redhat.com}{http://www.redhat.com}). Other
Linux or Unix brands should work without problems. We also have installations on
Windows NT 4.0 and Windows 2000.

\section{Webserver}
The reference webserver for OpenCms is, of course, the excellent Apache webserver \\
(\rqhttp{http://www.apache.org}{http://www.apache.org}).
Apache is the clear market leader with more than 50\% market share,
it's used by IBM as webserver for their operating systems, and it's also Open Source and
free of charge. We have been using OpenCms with Apache since version 1.3.6 on Linux and Windows.
For Windows systems, the Microsoft IIS (\rqhttp{http://www.microsoft.com/iis}{http://www.microsoft.com/iis})
has also been successfully tried with OpenCms. Other web servers should also be working without problems.

\section{Servlet runtime engine}
The Java Servlet API (JSDK) 2.2 and a servlet engine are needed by
the webserver to start up Java programs on http requests. The
servlet standard has been developed by SUN, see
\rqhttp{http://java.sun.com/products/servlet/index.html}{http://java.sun.com/products/servlet/index.html}.
We are most often using TOMCAT as a Java Servlets and Java Server
Pages reference implementation from the Apache Group. Please use
Tomcat 3.3 instead of 3.2.x. See
\rqhttp{http://jakarta.apache.org/}{http://jakarta.apache.org/}

For an Microsoft IIS installation, we have successfully used Jrun from Allaire \\
(\rqhttp{http://www.allaire.com/Products/Jrun/}{http://www.allaire.com/Products/Jrun/}).\\
Since servlets are a standard API which is part of the J2EE specification,
other servlets engines should be working without problems.

\section{Java VM}
To run servlets, a Java Virtual Machine (VM) is needed. Currently for development we are using
a version 1.3 compliant JDK, like the one provided \\
by SUN (\rqhttp{http://java.sun.com/products/jdk/1.3/}{http://java.sun.com/products/jdk/1.3/})\\
or IBM (\rqhttp{http://www.ibm.com/java/jdk/index.html}{http://www.ibm.com/java/jdk/index.html}).\\
For running systems we recommend using a Java VM version 1.3 or
higher.

\section{Database}
The OpenCms architecture allows to use any SQL capable database that offers a JDBC connector.
These include enterprise databases like e.g. Oracle (\rqhttp{http://www.oracle.com}{http://www.oracle.com}),
but also Open Source databases like MySQL (\rqhttp{http://www.mysql.com}{http://www.mysql.com}).

The reference installation uses MySQL. In order to access this database, the last pre-release
beta version of the mm JDBC driver is suggested\\
(\rqhttp{http://www.worldserver.com/mm.mysql}{http://www.worldserver.com/mm.mysql}. For
development currently we are using version version 3.23 of the database and version 2.0.1 of the
JDBC driver.

\section{XML Parser}
OpenCms makes heavy use of XML to store content and template data. For XML parsing, we make use of
the excellent Xerces developed by the Apache Group\\
(\rqhttp{http://xml.apache.org/xerces-j/index.html}{http://xml.apache.org/xerces-j/index.html}).
We recommend using version 1.2.3.

\section{Mail API}
The integrated task management is able to send mails when task events occur.
To enable this feature you will need the JavaMail 1.2 API available at \\
\rqhttp{http://www.javasoft.com/products/javamail/index.html}{http://www.javasoft.com/products/javamail/index.html}
 and the activation JavaBean
(\rqhttp{http://www.javasoft.com/beans/glasgow/jaf.html}{http://www.javasoft.com/beans/glasgow/jaf.html}).

\section{FESI (a Free EcmaScript Interpreter)}
OpenCms uses FESI as Interpreter for EcmaScript.
(\rqhttp{http://home.worldcom.ch/jmlugrin/fesi}{http://home.worldcom.ch/jmlugrin/fesi}).
