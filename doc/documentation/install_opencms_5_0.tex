\chapter{Installation of OpenCms 5.0}
\label{installation}

This chapter provides information on how to install OpenCms using
\rqhttp{http://jakarta.apache.org/tomcat/index.html}{Tomcat} and
\rqhttp{http://www.mysql.com/}{MySql}. All installation parts are
described as single steps. After completing each step you are
strongly advised to verify the success of your work.

\section{Install the Java JDK}
Install Java JDK 1.4 or later (from SUN
\rqhttp{http://java.sun.com/products/j2se/}{http://java.sun.com/products/j2se/}).
For details on how to install these components on your operating
system, see the documentation that comes with them.

\textbf{Important:} This version of OpenCms was tested with Java 1.4 only. 
Some features regarding file encoding where used that are not available with Java releases before 1.4.

\section{Install Tomcat}
OpenCms 5.0 requires a Servlet 2.3 / JSP 1.2 standards compliant container. 
Tomcat 4 is the reference implementation of this Standard. 
This release was tested with Tomcat 4.0.x and Tomcat 4.1.x. 
Older versions of Tomcat (3.x and earlier) do not support this newer standard and are thus not usable for OpenCms 5.0.

Install Tomcat from
\rqhttp{http://jakarta.apache.org/tomcat/index.html}{http://jakarta.apache.org/tomcat/index.html}
into a folder of your choice. This is the CATALINA\_HOME folder.
Don't forget to set the environment variables CATALINA\_HOME and
JAVA\_HOME.

Test the installation by running Tomcat in standalone mode and
check the examples. Note: Tomcat uses port 8080 in standalone
mode.
 
If you wish, you can combine the servlet-engine with a webserver
like the Apache Web Server
(\rqhttp{http://www.apache.org/httpd.html}{http://www.apache.org/httpd.html}).
Please see the documentation available with the webserver on how to combine it with your servlet environment.

\textbf{Important:} To make sure Tomcat works with the correct charset to read and
write files you should add the following parameter to the
commandline that is used to start Tomcat:

\texttt{-Dfile.encoding=ISO-8859-1}\\

You can also set the environment varibale CATALINA\_OPTS to do so:

(\texttt{CATALINA\_OPTS=-Dfile.encoding=ISO-8859-1})

The default encoding of OpenCms is \texttt{ISO-8859-1}, 
but you can also set another encoding supported by your Java VM, e.g. \texttt{UTF-8}. 
In case you do not care about the encoding, the default is most likely what you need.

\textbf{Setting the file encoding on Windows systems:} To avoid problems regarding the encoding, 
run Tomcat in standalone mode and edit the properties of the shortcut 
"Start Tomcat" in the start menu of Windows. Add the parameter \texttt{-Dfile.encoding=ISO-8859-1} to the command line behind the call of the java executable. 

Here's an example of how the start command of Tomcat should look like after the modification: 
\begin{verbatim}
c:\j2sdk1.4.1\bin\java.exe -jar -Dfile.encoding=ISO-8859-1 
-Duser.dir="C:\tomcat4" "C:\tomcat4\bin\bootstrap.jar" start
\end{verbatim}

\section{Install MySQL}
Install MySQL from
\rqhttp{http://www.mysql.com/downloads/index.html}{http://www.mysql.com/downloads/index.html}
(see the MySQL documentation on
\rqhttp{http://www.mysql.com/documentation/index.html}{http://www.mysql.com/documentation/index.html}).
This release was tested with MySQL 3.23.x and 4.0.x.

Note: On Windows-based systems MySQL has to be installed on the
\texttt{C:} drive and should be registered as service using
\texttt{(your MySQL path)/mysql/bin/mysqld -install}.

Start the MySQL server by running the service (WIN32) or executing
\texttt{(your MySQL path)/mysql/bin/mysqld} (UNIX).

Check that MySQL is running before you continue by starting the
MySQL monitor (execute \texttt{mysql} in your MySQL bin folder).
The database works correctly if a MySQL prompt appears after
calling the monitor. Quit the MySQL monitor by typing
\texttt{exit} and go to the next step.

\section{Deploy the opencms.war file}
Copy the \texttt{opencms.war} file from the binary distribution ZIP file to
\texttt{CATALINA\_HOME/web-apps/}. Replace
\texttt{CATALINA\_HOME} with the real path to your Tomcat
installation.

Start (or restart) Tomcat. Tomcat will now deploy the web application
OpenCms.

\textbf{Important:} OpenCms requires that it's \texttt{*.war} file is unpacked. 
OpenCms can not be deployed as war file only. 
Make sure Tomcat does unpack the war file and creates the \texttt{CATALINA\_HOME/webapps/opencms/} directory, 
placing the OpenCms files in this directory. 
The default configuration for your Servlet containers / environment could be to not unpack the deployed \texttt{*.war} file. 
If this is the case you must unpack the \texttt{opencms.war} file manually. 
Use an unzip tool for this, \texttt{*.war} files are just \texttt{*.zip} files with a different extension. 
The OpenCms setup wizard will display a warning and not allow you to continue 
if you did not unpack the  \texttt{*.war} file.

\section{Install OpenCms using the Setup-Wizard}
Start the Setup-Wizard by pointing your webbrowser to\\
\rqhttp{http://localhost:8080/opencms/ocsetup}
    {http://localhost:8080/opencms/ocsetup}.
Depending on your configuration, you have to replace \texttt{localhost} with your servername. 
The port 8080 is only used if you start Tomcat in standalone mode.

Follow the instructions of the OpenCms Setup-Wizard, using the "Standard" setup. It will set
up the OpenCms database and import all workplace resources into
the system. For normal installations with MySql and Tomcat running
on the same server all default settings will fit your needs.

\textbf{Important:} In case you want to import content from older OpenCms versions (5.0 beta 1 or 4.x), 
you must turn on the "directory translation" feature to make sure your pages still work. 
You can later change this setting (and all other settings selected during setup) 
by editing the file \texttt{CATALINA\_HOME/webapps/opencms/WEB-INF/config/opencms.properties}.

\section{Now your system is ready}
Now your system is ready to use. You can login with username:
\texttt{Admin} and password: \texttt{admin}. Please change this
password as soon as possible. The login URL of OpenCms in a default configuration is:\\
\rqhttp{http://localhost:8080/opencms/opencms/system/login/}
{http://localhost:8080/opencms/opencms/system/login/}

\section{Security issues}
After you have installed OpenCms you should have a look at the security settings.

First change the \texttt{Admin} password of OpenCms by calling the user
preferences (the "hammer" icon on the main screen of the workplace).
Then you should add a password to the MySQL database. Enter the following
commands at the MySQL command line.\\

\begin{quote}
\begin{verbatim}
use mysql;
insert into user values ('localhost', 'opencmsuser',password('XXXXX'),
  'N','N','N','N','N','N','N','N','N','N','N','N','N','N');
insert into db values ('localhost', 'opencms', 'opencmsuser', 'Y','Y',
  'Y','Y','Y','Y','Y','Y','Y','Y');
flush privileges;
\end{verbatim}
\end{quote}

Make sure you replace \texttt{opencmsuser} and \texttt{opencms} with the name of your user and database 
in case you have changed them on the setup wizard.

Don't forget to add the new user and password to all connect
strings of the database in your \texttt{opencms.properties} file. 
Only the new user can now connect to the OpenCms tables. For more
information see the MySQL documentation.
