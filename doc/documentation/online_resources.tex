\chapter{Useful resources}

\section{OpenCms online resources}

\begin{itemize}
\item Use the OpenCms mailinglist archive to search the {\tt opencms-dev} mailing list archives.
In case you have trouble setting up OpenCms, or in case you think you have found a bug,
make sure you check the archive first to see if your issue has already been addressed.
The archive is available at:\\
{\tt http://www.opencms.org/opencms/en/development/mailinglist-archive.html}
\item Please use our OpenCms bugzilla to report bugs that you have found in OpenCms. 
Before doing so, please check if the bug that you have found is certainly a new bug:\\
{\tt http://www.opencms.org/bugzilla/}
\item The CVS web interface allows you to browse through the OpenCms sources:\\
{\tt http://www.opencms.org/cvs}
\end{itemize}

\section{How you can help}
{\bf Please report bugs}\\
You might find that someting does not work as it should. If so, you should provide a bug report. 
Please use the OpenCms Bugzilla for all of your bug reports. You might have to create a Bugzilla 
account first. Please use the Bugzilla if possible and not the "opencms-dev" mailing list to report 
bugs. If you encounter setup related problems, please post them on the mailing list first, because 
in 99.5\% of all cases setup issues are not bugs in OpenCms but related to your local environment
configuration.\\
\\
{\bf Test out the new functionality}\\
The easiest way to participate in the development process would be to help testing the JSP integration, 
the interactive documentation and other new functionality. Please use the opencms-dev mailing list for 
discussions on the development process, about this documentation or about OpenCms in general. Note: This 
mailing list requires subscription prior to posting (see below). Please use the "opencms-dev" mailing list 
for questions and comments regarding OpenCms or the documentation and not Bugzilla.\\
\\
{\bf Provide examples and demos}\\
Another way to help extending OpenCms is to provide examples and demos for the use of JSP pages in OpenCms. 
If you contribute these, we might include these in this documentation in a later release. Or we could make 
your examples available as a separate module download. The best way to distribute your new demo or example 
would be to post it to the opencms-dev mailing list or to {\tt contributions@opencms.org}.

\section{How to subscribe to the opencms-dev mailing list}
The main place for the latest OpenCms related information is the mailing list\\ {\tt opencms-dev@opencms.com}. 
This list is for all general and development related news, messages and questions regarding OpenCms.
Traffic on the list is currently about 10 messages a day (as of February 2003).

To subscribe to the list, go to the URL:

{\tt http://www.opencms.org/opencms/en/development/mailinglist.html}

and just enter your email address and a password in the subscription form.
You have the possibility to change your subscription details later (set options like digest and delivery modes, get a reminder of your password, or unsubscribe from opencms-dev)
with your email address and chosen password.

Before posting a question to the list, you should check out the opencms-dev mailing list archive if this question 
has already been answered.