%
%  html2latex.tex
%
%  This is a support file for html2latex that defines a few items
%  necessary for HTML mode in latex
%
%%%%%%%%%%%%%%%%%%%%%%%%%%%%%%%%%%%%%%%%%%%%%%%%%%%%%%%%%%%%%%%%%%%%%%%%%%%
%
%  Update History:
%
%  who    when		what
%  ----  ------        --------------------------------
%  schaefer 8/97        Added overset and underset
%
%  dpvc  10/95		Wrote it.
%
%%%%%%%%%%%%%%%%%%%%%%%%%%%%%%%%%%%%%%%%%%%%%%%%%%%%%%%%%%%%%%%%%%%%%%%%%%%%
%%%%%%%%%%%%%%%%%%%%%%%%%%%%%%%%%%%%%%%%%%%%%%%%%%%%%%%%%%%%%%%%%%%%%%%%%%%%
%%%%%%%%%%%%%%%%%%%%%%%%%%%%%%%%%%%%%%%%%%%%%%%%%%%%%%%%%%%%%%%%%%%%%%%%%%%%
%
%  The following was the file html2tex-common.tex
%
%%%%%%%%%%%%%%%%%%%%%%%%%%%%%%%%%%%%%%%%%%%%%%%%%%%%%%%%%%%%%%%%%%%%%%%%%%%
%
%  Default is no indenting for paragraphs, and paragraphs separated by
%  a blank line.  We increase the stretchability of paragraph breaks a
%  bit, and require a bit more space between lines (this helps forms)
%
% changed
%\parindent=0pt
%\parskip=\baselineskip
%\advance\parskip by 0pt plus 6pt minus 3pt
%\multiply\lineskip by 2
%\multiply\normallineskip by 2

%
%  Some special characters need special treatment:
%
\def\htmlGt{$>$}
\def\htmlLt{$<$}
\def\htmlBar{$|$}
\def\htmlHat{{\char"5E}}
\def\htmlTilde{$\sim$}
\def\htmlBackslash{$\backslash$}

%
%  This routine is used by &frac12; and the realted entities
%
\def\htmlFrac#1/#2%           from exercise 11.6, the TeXbook
{%
  \leavevmode\kern.1em
  \raise .5ex\hbox{\the\scriptfont0 #1}\kern-.1em
  /\kern-.1em \lower.25ex\hbox{\the\scriptfont0 #2}%
}

%
%  In PRE-formatted mode, spaces will really be spaces
%
{\obeyspaces\gdef\htmlSpaces{\let =\ }}


%
%  We treat line breaks a bit carefully, because we want to be able to
%  remove them before <HR> in particular.  In order to do this, we
%  leave a special mark after a forced line break so we can recognize
%  it later.  To remove the break, we look for the special marker, and
%  if it is there, remove the marker and the line break.
%
\newdimen\htmlBRmark \htmlBRmark=1995sp
\def\htmlBR{\leavevmode \hfill\break\null\kern-\htmlBRmark\kern\htmlBRmark}
\def\htmlRemoveBR
{%
  \ifhmode
    \ifdim\lastkern=\htmlBRmark
      \unkern \unkern \setbox0=\lastbox \unpenalty \unskip
    \fi
  \fi
}

%
%  This routine allows us to insert postscript graphics with the
%  correct alignment (top, middle, bottom).  
%
\def\epsFile#1#2%
{%
  \leavevmode
  \ifx #1b\hbox{\epsBox{#2}}\else
  \ifx #1t\hbox{\eps@Top{\epsBox{#2}}}\else
  \ifx #1m\hbox{$\vcenter{\epsBox{#2}}$}\fi\fi\fi
}
%
%  Lower the contents of a box so that its top matches the top of the
%  line of text.
%
\def\eps@Top#1%
{%
  \setbox0=\vbox{#1}%
  \dimen0=\ht0 \advance\dimen0 by -\ht\strutbox
  \lower\dimen0\box0
}
%
%  You can redefine this to do whatever you want with the postscript
%  file name.
%
\def\epsBox{\epsfbox}


%%%%%%%%%%%%%%%%%%%%%%%%%%%%%%%%%%%%%%%%%%%%%%%%%%%%%%%%%%%%%%%%%%%%%%
%
%  Routines for handle forms
%

%
%  To format a text field, set \tt font, and get the width of a space.
%  Multiply this by the number of characters in the field and add two
%  more (I suppose we could have used em's here throughout, oh well).
%  Draw the underline (slightly below the baseline) and the add a
%  box of the correct width that contains the default data for the
%  field
%
\def\htmlInputText#1#2%
{{%
  \tt\setbox0=\hbox{ }%
  \dimen0=#1\wd0 \advance\dimen0 by 2ex
  \lower .4ex \hbox to 0pt{\vbox{\hrule width \dimen0}\hss}%
  \hbox to \dimen0 {\kern 1ex #2\hss}%
}}

%
%  For a checkbox, make sure we're in horizontal mode, then make a box
%  around the check bullet mark (if this box is checked) or around a
%  blank box of the same size as the check bullet mark.  Raise the box
%  a bit so that it is above the baseline slightly.
%
\def\htmlInputCheckbox#1%
{%
  \leavevmode
  \raise .1em \hbox
  {%
    \htmlBox{.2em}%
    {%
       \ifx\empty#1\empty \phantom{\htmlCheckBullet}%
	 \else \htmlCheckBullet \fi
    }%
  }%
}
%
%  The check bullet mark is a solid square
%
\def\htmlCheckBullet{\kern-.2em\vrule width .4em height .4em\kern-.2em}


%
%  For a radio button, either use the big empty circle, or use the big
%  cirlce with a bullet in the center.
%
\def\htmlInputRadio#1%
  {\ifx\empty#1\empty $\bigcirc$\else
     \htmlOverlay\bigcirc{\strut\lower .2ex\hbox{$\bullet$}}\fi}

%
%  A button is a box around another box around the name of the button
%
\def\htmlInputButton#1{\htmlBox{1pt}{\htmlBox{.5ex}{\strut#1}}}


%
%  A pull-down menu is a button with the given name and a big triangle
%  pointing down at the end
%
\def\htmlInputMenu#1%
{
   \htmlInputButton
     {\ignorespaces#1\unskip\ \raise .3ex\hbox{$\bigtriangledown$}}%
}

%
%  The select marker for a multi-line menu item is a bullet
%
\def\htmlSelectBullet{\leavevmode\hbox to 1em{\hss$\bullet$\hss}}

%
%  A multi-line menu is just a multi-line text area
%
\let\htmlSelect=\htmlTextarea

%
%  A multi-line text area obeys spaces and lines
%  It is typset in \tt and is a box that is the specified number of
%  characters tall and wide.  There is no paragraph spaces or indents
%  in the typein area.  Once the box is typeset, the correct amount of
%  material is split off the top of it, and it is placed in a centered
%  box with an outline
%  
\def\htmlTextarea{\bgroup \obeylines\obeyspaces\htmlSpaces \htmlDoTextarea}
\def\htmlDoTextarea#1#2#3\htmlEnd
{%
  \tt\setbox0=\hbox{ }%
  \dimen0=#1\wd0
  \dimen1=#2\baselineskip
  \setbox1=\vtop
  {%
    \parskip=0pt \parindent=0pt
    \hsize=\dimen0
    \noindent\strut #3\vfill
  }%
  \splittopskip=\topskip
  \vbadness=10000
  \setbox2=\vsplit1 to \dimen1
  \htmlBox{.5ex}{$\vcenter to\dimen1{\unvbox2}$}%
  \egroup
}


%
%  This routine draws a boder around a box.  The border is separated
%  from the contents of the box by the amount of space specified in
%  #1.  The width of the line is the standard .4pt.  The baseline of
%  the constructed box should be the baseline of the material
%  containined within it.
%
\def\htmlBox#1#2%
{{%
  \setbox0=\hbox{#2}%
  \dimen1=#1
  \dimen0=\dp0 \advance\dimen0 by \dimen1 \advance\dimen0 by .4pt
  \leavevmode \lower \dimen0 \hbox
  {%
    \vrule
    \vbox
    {%
      \hrule\kern\dimen1
      \hbox{\kern2\dimen1 \box0 \kern2\dimen1}%
      \kern\dimen1\hrule
    }%
    \vrule
  }%
}}

%
%  This routine is used to contruct special symbols like the checkbox
%  and radio button.  It centers one symbol on top of another (the
%  larger one is given first, and both are assumed to be in math
%  mode).
%
\def\htmlOverlay#1#2%
{{%
  \setbox0=\hbox{$#1$}%
  \hbox
  {%
    \vbox to \ht0 {\vss \hbox to \wd0 {\hss$#2$\hss}\vss}%
    \kern-\wd0\box0
  }%
}}

%%%%%%%%% End of the old file html2tex-common.tag %%%%%%%%%%%%%%%%%%%%%%%%%%
%%%%%%%%%%%%%%%%%%%%%%%%%%%%%%%%%%%%%%%%%%%%%%%%%%%%%%%%%%%%%%%%%%%%%%%%%%%%
%%%%%%%%%%%%%%%%%%%%%%%%%%%%%%%%%%%%%%%%%%%%%%%%%%%%%%%%%%%%%%%%%%%%%%%%%%%%
%%%%%%%%%%%%%%%%%%%%%%%%%%%%%%%%%%%%%%%%%%%%%%%%%%%%%%%%%%%%%%%%%%%%%%%%%%%%




%
%  Handle a <DD> that is not preceeded by <DT>
%
\def\htmlDD{\item \leavevmode \kern-\itemindent \kern-\labelsep}

%
%  Do a <HR>:
%    Remove a previsous <BR> if any
%    If we're in horizontal mode, end it
%    Otherwise if there is no previuous skip, add the paragraph skip
%     (this is a hack, since we can't tell whether there was a
%      preceeding <P> or not, so we assume there was)
%    Add the rule
%    Undo the parskip that will be added by the next paragraph
\def\htmlHR
{%
  \htmlRemoveBR \ifhmode \endgraf \else
     \ifdim\lastskip=0pt \vskip\parskip \fi\fi
  \kern6pt \hrule width\hsize height 1pt \kern6pt
  \vskip-\parskip
}

%
%  Define a PRE environment for preformatted text
%
%  Starting PRE:
%    Ends the paragraph and inserts parskip (since we set it to 0 below)
%    Sets the paragraph skipping to 0
%    Obeys lines and spaces, and makes spaces be real spaces
%    Sets \tt font
%
%  Ending PRE:
%    Ends the paragraph (the following paragraph will add the parskip
%      space)
\newenvironment{PRE}{\htmlPRE}{\htmlEndPRE}
\def\htmlPRE
{%
  \endgraf\vskip\parskip
  \parskip=0pt
  \obeylines\obeyspaces\htmlSpaces\tt
}
\def\htmlEndPRE{\endgraf}


%%%%%%%%%%%%Definitions added by J. Schaefer%%%%%%%%%%%%%%%%%%%%%%%%%%%%%%%%

% \htmlOVERSET
%
% Set expression #1 directly above expression #2:
%   raise by height = (height of #2)+(0.5ex)+(depth of #1)
%   To center #1 relative to #2, put #1 in a box roughly the 
%     width of #2 (automatically centered)
%   Raise #1's box
%   Pretend #1's box has 0 width, so that #2 can be placed below it.
%
\newlength{\HTMLBasewd}
\newlength{\HTMLBaseht} \newlength{\HTMLUpperdp}
\newcommand{\htmlOVERSET}[2]{%
\settoheight{\HTMLBaseht}{$#2$}%
\addtolength{\HTMLBaseht}{0.5ex}%
\settodepth{\HTMLUpperdp}{$#1$}%
\addtolength{\HTMLBaseht}{\HTMLUpperdp}%
\settodepth{\HTMLUpperdp}{$#1$}%
\settowidth{\HTMLBasewd}{$#2$}%
\raisebox{\HTMLBaseht}{\makebox[0pt]{\hspace{\HTMLBasewd}$#1$}}%
{#2}}%

%
% \htmlUNDERSET
%
% Set expression #1 directly below expression #2:
%   lower by amount = (depth of #2)+(0.5ex)+(height of #1)
%   To center #1 relative to #2, put #1 in a box roughly the 
%     width of #2 (automatically centered)
%   Lower #1's box
%   Pretend #1's box has 0 width, so that #2 can be placed above it.
%
\newlength{\HTMLLowerht} \newlength{\HTMLBasedp}
\newcommand{\htmlUNDERSET}[2]{%
\settoheight{\HTMLLowerht}{$#1$}%
\settodepth{\HTMLBasedp}{$#2$}%
\addtolength{\HTMLLowerht}{0.5ex}%
\addtolength{\HTMLLowerht}{\HTMLBasedp}%
\settowidth{\HTMLBasewd}{$#2$}%
\raisebox{-\HTMLLowerht}{\makebox[0pt]{\hspace{\HTMLBasewd}$#1$}}%
{#2}}%


