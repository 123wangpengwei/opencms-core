\chapter{Core Development}

\section{Introduction}

In this section you can learn something about the interna of OpenCms. This
is interesting for people who want to contribute to the project at 
core\-level. Currently there are two subsections. One about how to build
OpenCms from scratch by your own using ant. The other about interna of the
project mechanism used by OpenCms since version 4.4.

\section{How to compile the OpenCms core}
OpenCms provides a full source distribution that you can use to build the
OpenCms core. This is only needed if you want to add your own core 
extensions. To develop the usual kind of website functionality, you 
should use the OpenCms module mechanism which is much easier to start with 
and also much better documented. Before even considering starting to work 
on the core, you definitly should have written some OpenCms modules to 
understand the separation between a module and a core extension. 

The OpenCms core comes with the best possible documentation: The source
code itself. This is really not something for the novice Java developer.
However, if you have some experience in Java, Java Servlets, JDBC, XML in 
general and the Apache Xerces API you might take a look. As said before, 
you should also have already a firm understanding of the OpenCms module API. 

Since this is rather deep stuff more for experts than for beginners, I 
will not explain every detail of the process.

\subsection{Before you start}

In case you really intend to contribute to the OpenCms core development, 
please make sure you use the latest version from the CVS source tree. The
source distribution ZIP file provided for download is the code of the latest
major release, not the current development status of OpenCms. The ZIP provides
a good starting point, but contributions should be based on the latest 
development version. The way to do it is to first get the source distribution 
(because it also contains the libaries), see if you can build this, and then 
simply checkout the latest \texttt{/opencms} directory from the CVS, replacing the 
\texttt{/opencms} directory from the source distribution (see below).

Ok, now on to the real work:

\subsection{Have Ant installed}

Apache Ant is a Java based build tool. In theory it is kind of like make
without make's wrinkles. You need Ant version 1.4 or later to build the 
OpenCms core. Ant is part of the Jakarta Apache Project 
(\rqhttp{http://jakarta.apache.org/}{http://jakarta.apache.org/}) and can be downloaded
here: \rqhttp{http://jakarta.apache.org/ant/}{http://jakarta.apache.org/ant/}. 
Please check the Ant documentation to make sure you understand the basic 
principles behind Ant. 

Ant installation is described in the manual
(\rqhttp{http://jakarta.apache.org/ant/manual/}{http://jakarta.apache.org/ant/manual/}). 
It requires that you have set up your path to Java correctly. Make sure 
Ant runs before proceeding.

\subsection{Get the OpenCms Source distribution}

Download the latest OpenCms source distribution. It contains all classes 
neccessary to build the OpenCms core. You can also checkout an OpenCms version
from our CVS tree, but make sure you have downloaded the source distribution 
before that because the necessary libraries are not included in the CVS. 

The source distribution comes in a ZIP file. Unpack this in your working 
directory, lets say this is called \texttt{/work}. You will then end up with the 
following structure in your work directory:

\begin{verbatim}
/work
    /opencms               <= Start of the OpenCms source tree
        /etc
        /src
        /web
        build.xml          <= This is needed by Ant 
        ...  
    /ExternalComponents    <= Contains the neccessary libaries
        activation.jar
        fesi.jar
        ...
\end{verbatim}


\subsection{Get additional classes (optional)}

In case you want to use the Oracle database, you need the Oracle JDBC driver
which you can download from the Oracle Technology Network 
(\rqhttp{http://otn.oracle.com}{http://otn.oracle.com}). You have to register 
there to get the driver. The file 
you need is the JDBC driver called classes12.zip. Place the file in the 
directory \texttt{/ExternalComponents} of the OpenCms source tree (see above). If you 
don't have this file, the OpenCms Oracle Driver 
(package org.opencms.file.oracleplsql) is not compiled, which is ok if you 
don't run on Oracle.

\subsection{Build the source by starting Ant}

Ok this is the easy part. Call up a commandline, move to the \texttt{/opencms} 
directory (where the file \texttt{build.xml} resides) of the OpenCms source tree. 
In your commandline, enter the following: 

\texttt{ant all} 

That's it! This will build a complete OpenCms distribution. Your work 
directory will look like this after Ant is finished: 

\begin{verbatim}
/work
    /opencms               <= Unchanged
    /ExternalComponents    <= Unchanged
    /build
        /classes           <= Will contain the compiled OpenCms classes
        /opencms           <= This is the standard .war directory layout
            /WEB-INF
                /lib
                /oclib
            /META-INF
    /zip                   <= The OpenCms distribution file will be placed
    /pdf                      here
\end{verbatim}


The final result of the compilation will be a ZIP file which will be placed 
in the \texttt{/zip} directory. This ZIP is exactly the same layout as the OpenCms
binary distributions, so it will contain the opencms.war archive. 

\subsection{Install your new version}

Now that you have your new OpenCms binary distribution, you simply need to 
follow the installation guide for your server setup. Please make sure you 
don't mess up any existing installation. Best have a separate machine that 
is used only for testing and development. 

\subsection{Other Ant targets}

There are several more targets in the ANT that might be useful for you.
Here's a short overview:

\begin{itemize}
\item Ant target: \texttt{war}\\
This creates the opencms.war, but does not create the binary distribution 
ZIP file.

\item Ant target: \texttt{srcdist}\\
This generates the source distribution ZIP file. Like the one you 
can download. 

\item Ant target: \texttt{tomcat.dist}\\
This is useful if you really are in developing core extensions. Provided
that OpenCms runs on your machine using Tomcat (and that you have all 
environment variables set up, esp. TOMCAT\_HOME or CATALINA\_HOME), calling 
this target will have Ant updating the OpenCms classes on your machine. 
This is done by replacing the files in the Tomcat \texttt{webapps/opencms} directory.
If you have renamed opencms.war, or if your Tomcat is not installed the 
usual way, this probably will not work.

\item Ant target: \texttt{setup}\\
This is an extension to tomcat.dist: It depends on tomcat.dist, but will 
also update the OpenCms database, provided you run MySQL on your development
machine. This is usefull if you do changes to the OpenCms Workplace template
files. CAUTION: This drops your complete database without further notice. 
Don't use until you know what you do. 

\item Ant target: \texttt{tex}\\
This generates the OpenCms documentation from the TeX sources. It's outside 
the scope of this document to explain what setup you need to do that.

\end{itemize}

To get a complete actual list of all targets please call ant with this parameter:\\
\texttt{ant -projecthelp}.

\subsection{Important}

In case you replace files in the Tomcat OpenCms \texttt{webapps} directory without doing a new setup of the database, you will get a message on login saying something like: 

\texttt{The version of workplace-templates and workplace-classes are different.\\Please update one.}

Since the workplace templates make very close usage of the core, this should 
definitely be fixed on production systems. To remove the message, make a new 
setup of the OpenCms database using the installation wizard from your newly 
generated binary distribution file. If you know what you do, you can ignore 
this message on development machines. 
