\chapter{\label{42install}Installation of OpenCms}
%-ltoh-   title := "OpenCms 4.2 - Installation"
This chapter provides information on how to install OpenCms using
all standard components described in the previous chapter. All installation
parts are described as single steps. After completing a step you are strongly
advised to verify the success by running the particular commands and 
watching the results.

If not mentioned, all commands refer both to UNIX and Windows systems.
All shell commands containing path information are written in UNIX style using
a slash "/" as folder separator. On Windows you will have to
replace these characters by a backslash "$\backslash$". 
Be very careful with replacing this characters since the virtual filesystem
of OpenCms uses a slash. Using a backslash in virtual filenames will 
prevent OpenCms from running properly! Ensure that you really only replace
slashes in shell commands and paths concerning the real file system.

You also will need a 
MS-DOS prompt in order to execute all commands in a Windows enviroment.

\section{\label{42apache}Installing the Apache Web Server}

Install the Apache Web Server \rqhttp{http://www.apache.org/httpd.html}{http://www.apache.org/httpd.html}. 
Write down the document root path of the Apache containing all static HTML 
pages, your will need it later for configuring OpenCms.
On WIN32 systems we recommend installing the Apache as service
using \texttt{apache.exe -i}. 

Run your web server by calling the \texttt{httpd} deamon in the Apache
\texttt{bin} folder (UNIX)
or starting the service (Windows). Check that it
is working by requesting an example page 
(e.g. the demo \texttt{index.html} that comes with the Apache distribution).


\section{\label{42jdk}Installing Java JDK and JSDK}
Install the Java JDK from SUN 
(\rqhttp{http://java.sun.com/products/j2se/}{http://java.sun.com/products/j2se/})\\
or IBM (\rqhttp{http://www3.software.ibm.com}{http://www3.software.ibm.com}) \\
and the JSDK 
(\rqhttp{http://java.sun.com/products/servlet/}{http://java.sun.com/products/servlet/}).

After this, install the servlet engine Apache Jserv 1.1.2 from \\
\rqhttp{http://java.apache.org/jserv/index.html}{http://java.apache.org/jserv/index.html}.

For details on how to install these components on your operating system,
see the documentation that comes with them.

Write down your JServ installation path. Like the Apache document root, you
will need it later.

Make sure that you update the \texttt{PATH} environment variable with JDK \texttt{bin}
directory, e.g. \texttt{/usr/java/bin} on UNIX and \texttt{C:$\backslash$\backbug{jdk}$\backslash$\backbug{bin}}
on Windows systems.

Check that a JServ servlet zone is set up properly and running on your 
machine by requesting the default
http://your server/servlets/IsItWorking class that comes with JServ. 

\section{\label{42mysql}Installing MySQL}
Install MySQL from \rqhttp{http://www.mysql.com/downloads/index.html}{http://www.mysql.com/downloads/index.html}
(see the MySQL documentation on \rqhttp{http://www.mysql.com/documentation/index.html}{http://www.mysql.com/documentation/index.html}). 

Please remember that MySQL 
is shareware and that you have to pay for it if you use it for a commercial purpose.

Note: On Windows-based systems MySQL has to be installed on the \texttt{C:$\backslash$} drive
and should be registered as service using \texttt{(your MySQL path)/mysql/bin/mysqld -install}.

Start the MySQL server by running the service (WIN32) or executing 
\texttt{(your MySQL path)/mysql/bin/mysqld} (UNIX).

Check that MySQL is running before you continue by starting the MySQL monitor 
(command \texttt{mysql} in your MySQL \texttt{bin} folder).
The database works correctly if a MySQL prompt appears after calling the monitor. 
Quit the MySQL monitor by typing \texttt{exit} and go to the next step. 

\section{Installing the OpenCms files}
We currently do not have an RPM or setup.exe, 
but there are plans to make them available in the near future.
We recommend that you unpack your OpenCms distribution archive file 
(opencms\_4.x.yy from \\
\rqhttp{http://www.opencms.com/servlets/opencms/service/download/index.html}
{http://www.opencms.com/servlets/opencms/service/download/index.html})
in your \texttt{C:$\backslash$\backbug{Program Files}$\backslash$} folder on Windows systems and 
in the \texttt{/opt} folder on UNIX systems. 

While extracting the ZIP archive
a new folder \texttt{opencms\_4.x.yy} (with \texttt{x} replaced by the release and \texttt{yy}
replaced by the current build number) will be created. This new folder will be your OpenCms home
directory. Make it to your current directory and create a subfolder lib.
This folder will be the place for all external components and drivers used by OpenCms. 
On UNIX systems take care that OpenCms has the access rights to read the content
of the folder. In order to run OpenCms properly
you will need the following JAR files:

\begin{itemize}
\item \texttt{xerces.jar} from the Apache Xerces XML parser. 
\item The MySQL JDBC database driver \texttt{mm.mysql-2.0.1-bin.jar}.
\item \texttt{mail.jar} and \texttt{activation.jar} from the JavaMail distribution.
\item \texttt{fop\_bin.jar} from the FOP packet of the Apache XML project.
\end{itemize}

See the Components page on the OpenCms web site \\
(\rqhttp{http://www.opencms.com/servlets/opencms/background/components.html}
{http://www.opencms.com/servlets/opencms/background/components.html}) 
or the chapter \ref{components} about OpenCms components in the OpenCms documentation 
for a more detailed description of these external parts of OpenCms.


\section{\label{42dbsetup}Setting up the OpenCms database}
For setting up the MySQL database for OpenCms, create a new database \texttt{opencms42}
by calling

\begin{quote}
\texttt{(your MySQL path)/mysql/bin/mysqladmin create opencms42}
\end{quote}

It is important to name the database \texttt{opencms42} because the configuration file of
OpenCms (\texttt{opencms.properties}) creates a database connection to a database named \texttt{opencms42}.
If you want to use another name for your database
you will have to change the line for the database connection in the \texttt{opencms.properties} file.

Run
\begin{quote}
\texttt{(your MySQL path)/mysql/bin/mysql opencms42 < config/mysql.sql}
\end{quote}
This will create all of the required database tables. 
Make sure that all of the tables are created properly by calling the MySQL monitor again with

\begin{quote}
\texttt{(your MySQL path)/mysql/bin/mysql opencms42}
\end{quote}

and entering 
\begin{quote}
\texttt{show tables;}
\end{quote}
at the monitor prompt. MySQL should print the following list of tables: 

\begin{quote}
\begin{verbatim}
+--------------------------+
| Tables_in_opencms42      |
+--------------------------+
| cms_files                |
| cms_groups               |
| cms_groupusers           |
| cms_projects             |
| cms_properties           |
| cms_propertydef          |
| cms_resources            |
| cms_sessions             |
| cms_systemid             |
| cms_systemproperties     |
| cms_task                 |
| cms_tasklog              |
| cms_taskpar              |
| cms_tasktype             |
| cms_users                |
+--------------------------+
\end{verbatim}
\end{quote}

On success, you now can exit MySQL simply by typing
\begin{quote}
\texttt{exit}
\end{quote}

\section{Initializing the OpenCms database}
Edit the \texttt{opencms.properties} file in the config folder of your OpenCms home 
directory and set the path for the file containing the registry:

\begin{quote}
\texttt{registry = (your OpenCms home directory)/config/registry.xml}
\end{quote}

Also change the entry for your log file:

\begin{quote}
\texttt{log.file=(your OpenCms home directory)/logs/opencms.log}
\end{quote}

If not already existing, you have to create the logs folder where the opencms.log file 
will be located. On UNIX systems, Apache may have problems writing to the logs folder. 
For this reason you should set its owner to nobody. 

OpenCms is able to export folders to the real file system on the server during 
publishing operations. This feature commonly is used for storing pictures or large downloadable
binary files in the Apache Web Server's
document root. This will increase performance while displaying web pages with OpenCms.
We recommend to set the following export paths in the \texttt{opencms.properties} file:

\begin{quote}
\begin{verbatim}
exportpoint.0 = /pics/
exportpoint.path.0 = (real path to apache document root)/pics/
exportpoint.1 = /download/
exportpoint.path.1 = (real path to apache document root)/download/
exportpoint.2 = /system/workplace/pics/
exportpoint.path.2 = (real path to apache document root)/pics/system/
\end{verbatim}
\end{quote}

Note the trailing slash
characters. They are very important and must not be omitted.

Ensure that all paths you are using as export points in your real file system
do really exist. OpenCms won't try to create these folders.

Similar to the log file on UNIX systems access rights conflicts may occur while
OpenCms is trying to write to these folders. Set their owner to nobody, too.

Now take a look at the file \texttt{cmssetup.txt}:
Set the path for OpenCms export files appropriate to your system by editing
the line beginning with "\texttt{writeExportPath}" to 

\begin{quote}
\texttt{writeExportPath "(your OpenCms home directory)/export/"}
\end{quote}

(you will find these lines near to the end of the file).
Again the trailing slash is very important.
If not already existing you have to create the export folder in your OpenCms home
directory. On UNIX sytems ensure that OpenCms can access this folder
(the best way to ensure this is setting the owner to nobody and give
the owner rwx rights). 

Make the OpenCms home directory to your current directory again and
initialize the database with all workplace default files by calling the following command:

\begin{quote}
\texttt{java -mx64M \\
-classpath opencms.jar:lib/xerces.jar:lib/mm.mysql-2.0.1-bin.jar\\
com.opencms.core.CmsShell\\
config/opencms.properties < config/cmssetup.txt}
\end{quote}

Note: This must be written in one single command line! The Java binary \texttt{java} should 
reside in the \texttt{bin} folder of your JDK installation. If you have added this folder to your 
system \texttt{PATH} environment as recommended during the installation of the Java VM,
this command should work without giving a path.

If you have a Java runtime environment 
on your system rather than a complete JDK, you
can replace the \texttt{java} command by \texttt{jre}.

Note: On Windows-based systems you have to replace the colons used as classpath separators 
by semicolons.

If there are errors, please carefully check the classpath option of your command.
Some Java VM need additional settings for the classpath. Try to add the current directory 
and (for JDK 1.1.x) the \texttt{classes.zip} in the \texttt{lib} folder of your JDK installation to your 
classpath, if the command above doesn't work and throws a \texttt{ClassNotFoundException} error message.
After fixing a problem, you should drop (\texttt{mysqladmin drop opencms42}) and recreate 
the existing database \texttt{opencms42} (see previous section) before 
running the command above for a second time.

After this the OpenCms database setup is completed. You are now able to check the basic OpenCms system. 
Call the OpenCms console by entering:

\begin{quote}
\texttt{java -mx64M\\
-classpath opencms.jar:lib/xerces.jar:lib/mm.mysql-2.0.1-bin.jar\\
com.opencms.core.CmsShell config/opencms.properties}
\end{quote}

This command is similar to the one above.
Note that the part "\texttt{< config/cmssetup.txt}" is missing now.

Log in as Admin and check the accessibility of a workplace file in your database. 
You execute these commands on the shell: 

\begin{quote}
\texttt{login Admin admin\\
readFile "/system/workplace/action/start.html"}
\end{quote}

The OpenCms console answers e.g. with

\begin{quote}
\begin{verbatim}
[Resource]:/system/workplace/action/start.html,
Project=1 , User=2 , Group=3 : Access=rwvr------ : Resource-type=2 : 
 Locked=-1 : lenght=168 : state=0
\end{verbatim}
\end{quote}

Quit the console by typing \texttt{exit}.

\section{Configuring your servlet zone}
Finalize the configuration of your servlet zone. Make sure that your \texttt{jserv.conf} in 
the \texttt{conf} folder of your Apache JServ installation directory contains the line

\begin{quote}
\begin{verbatim}
ApJServMount /servlets /root
\end{verbatim}
\end{quote}

Note: \texttt{ApJServMount}, \texttt{/servlets} and \texttt{/root} are separated by blanks.

The file \texttt{jserv.properties} in the JServ \texttt{conf} folder must contain

\begin{quote}
\texttt{zones=root\\
wrapper.bin.parameters=-mx64M\\
root.properties=(your servlet configuration path)/zone.properties}
\end{quote}

Note: On Windows systems you should use a backslash instead of a slash. 

Update your \texttt{zone.properties} and set the following repositories: 

\begin{quote}
\begin{verbatim}
repositories=(your OpenCms home directory)/opencms.jar
repositories=(your OpenCms home directory)/lib/xerces.jar
repositories=(your OpenCms home directory)/lib/mm.mysql-2.0.1-bin.jar
repositories=(your OpenCms home directory)/lib/mail.jar
repositories=(your OpenCms home directroy)/lib/activation.jar
repositories=(your OpenCms home directroy)/lib/fop_bin.jar
\end{verbatim}
\end{quote}

and add to the \texttt{Servlet Aliases} section the line: 

\begin{quote}
\texttt{servlet.opencms.code=com.opencms.core.OpenCmsServlet} 
\end{quote}

and to the \texttt{Aliased Servlet Init Parameters} section 

\begin{quote}
\texttt{servlet.opencms.initArgs=properties=(OpenCms home)/config/opencms.properties}
\end{quote}
Note: On Windows systems you have to use a backslash instead of a slash. 

After configuring your servlet zone you should restart the Apache to
ensure the new settings are read by the servlet engine.

\section{Now your system is ready}

Start your web server. The OpenCms system is available at\\
http://your.server/servlets/opencms/system/workplace/action/login.html.\\
Login as Admin using the default password admin. 

\section{Security}

After you have installed OpenCms you should look at the security.
Change the Admin password of OpenCms by calling the user preferences 
("hammer" icon on the main screen of the workplace). 
Then can add a password to the MySQL database. Enter the following commands 
at the MySQL command line.\\ 

\begin{quote}
\begin{verbatim}
use mysql;
insert into user values ('localhost', 'ocmsuser',
password('XXXXX'),'N','N','N','N','N','N','N','N','N',
'N','N','N','N','N'); 
insert into db values ('localhost', 'opencms42', 'ocmsuser',
'Y','Y','Y','Y','Y','Y','Y','Y','Y','Y'); 
flush privileges; 
\end{verbatim}
\end{quote}

Don't forget to add the new user and password to all connect strings 
of the database to your \texttt{opencms.properties}. 
The new user can now only connect to the OpenCms tables. 
For more information see the MySQL documentation. 

\section{Troubleshooting}

Look at all of the available log files 
(here with the paths used in a standard UNIX installation) 
if you have problems calling the login screen: 

\begin{itemize}
\item Apache log file (usually \texttt{(Apache home)/logs/error\_log})
\item JServ module log file (usually \texttt{((JServ home)/logs/mod\_jserv.log})
\item OpenCms log file (\texttt{(OpenCms home)/logs/opencms.log})
\end{itemize}

\section{Installation of the example news module}

The setup of the basic system is now complete, but you may want
to add some more functionality to it.
A good way to get used to the system is to practice with the 
module mechanism and to install a first module. 

A simple newsfeed module came with the basic system or can be downloaded
from the opencms homepage. The installation of the module is quite easy.

\begin{itemize}
\item Unzip the file \texttt{NewsModule.zip}.
The zip-file contains two files: another zip containinig the real module 
(\texttt{com.opencms.modules.homepage.news\_2.zip})
and a script to set up the MySQL database (\texttt{NewsModuleSetup.sql}).

\item
First prepare the database.
Change to the folder in which the file \\
\texttt{NewsModuleSetup.sql} resides and type in the following shell command:

\begin{quote}
\texttt{(your MySQL path)/bin/mysql opencms42 < NewsModuleSetup.sql}
\end{quote}

This creates all necessary tables and inserts some test data.

\item Then upload the module.
Move the file \\
\texttt{com.opencms.modules.homepage.news\_2.zip} to the \texttt{modules}
subdirectory in your export path.

Open the OpenCms workplace and change to the module management under the
administration view.
Make sure that you are within the online project.
Choose the icon to upload a new module.
Choose upload new module from server and continue.

The module should appear in the selectbox. 
Select \texttt{continue} to start uploading the module.
\end{itemize}

If all went well, the module is now installed.
The module provides two new administration points. A further documentation
can be found in the virtual filesystem. 

Start exploring under \texttt{/moduledemos/com.opencms.modules.homepage.news/}. 


\section{Setting up a local copy of the OpenCms homepage}
You can complete your test system by setting up a local copy of the OpenCms homepage.
The OpenCms homepage uses the news module, so it is necessary to install the module if
you want to make your first steps with the OpenCms homepage.

To get the templates of the OpenCms homepage you have to make a simple import.
The required import file (\textit{website\_import.zip}) is shipped with OpenCms 4.2,
together with another sql-script (\textit{website.sql}). This script inserts the content
which is handled by the News Module into the database.
Both files are bundled to the zip file \textit{website.zip}.

\begin{itemize}
\item
Extract this file first and then move the file \textit{website\_import.zip} to your
export path.

\item
Then insert the content into the database by using the sql-script.
Change to the folder in which the file
\texttt{website.sql} resides and type in the following shell command:

\begin{quote}
\texttt{(your MySQL path)/bin/mysql opencms42 < website.sql}
\end{quote}

\item
Then you have to import the templates.
To be able to do this, you have to be within a project
(not the online project), so create a new
project first. \\
Change to the administration view and choose \texttt{Projectmanagement}
-> \texttt{New Project}.\\ 
Specify a project name, for example \texttt{OpenCmsHomepage} and the folders that
should be included. \\
In this case include everything, type in \texttt{/} and press
the button with the blue arrow to adopt this directory to the list of folders that
have to be included.

After creating the new project, make sure that you are actually inside it and
change again to the administration view. 

\item
Now start the import of the OpenCms homepage. 
Make sure that the import file \\
\texttt{website\_import.zip} can be found within your
export directory.
Choose the point \texttt{Databasemanagement} and then \texttt{import}.
Choose the file \texttt{website\_import.zip} from the selectbox and start the process.

\end{itemize}
That's all.
Note: The pictures are missing right now, but they will be available after
publishing the project.

\section{Setting up OpenCms with different components}
OpenCms is developed and tested in the environment described in the
chapter \ref{components} about OpenCms components in the 
OpenCms documentation. Since the system is based on standards and
written 100\% in Java, it should be easy to use other components.

Global and detailed instructions for all possible constellations of components
would go out of the scope of this documentation. However, we are very
interested in descriptions on how to set up OpenCms with other
well-known and commonly used environments. 

If you managed to set up OpenCms with different components, please let us know. 
You can send your configuration information to documentation@opencms.com.

\subsection{Installation on Solaris with BEA Weblogic 5.1 and Oracle 8}
Heiko Kirschke has set up OpenCms on 
a Solaris system with a BEA Weblogic 5.1 application server using an
Oracle 8 database. He sent us the follwing annotions. All step numbers
refer to the corresponding steps in the instructions for the reference system.

If you have questions
concerning these instructions, please directly contact Heiko Kirschke 
(mailto:Heiko.Kirschke@acm.org).


\begin{itemize}

\item \textbf{Step 1:} Installing Apache is not necessary for BEA Weblogic,
since BEA WebLogic has its own Web server. Use this URL instead:
http://your.server:7001, this should come up with the standard BEA
page.

\item \textbf{Step 2:} For BEA WebLogic, install the J2EE additionally. Set up
an environment variable \texttt{J2EE\_HOME} which points towards this
installation directory.

Installing Apache JServ is not necessary for BEA
Weblogic, since BEA WebLogic has its own servlet engine.

\item \textbf{Step 3:} With BEA WebLogic, normally Oracle (or one of the other
major commercial RDBMs) is used. Haven't checked BEA WebLogic with
mySQL up to now.

\item \textbf{Step 4:} Assuming that \texttt{\$WL\_HOME} points towards the BEA WebLogic
installation directory (e.g. \texttt{\$WL\_HOME=/usr/local/weblogic}), create
directory\\
\texttt{\$WL\_HOME/myserver/servletclasses/opencms} and put these files
there. I prefer to use a symbolic link `\texttt{opencms}' here, with `\texttt{opencms}'
being a symbolic directory link to the current used OpenCms version.

The document root folder is \texttt{\$WL\_HOME/myserver/public\_html}

In BEA WebLogic, \texttt{(your servlet path)/servlets} becomes\\ 
\texttt{\$WL\_HOME/myserver/servletclasses/opencms}, i.e.:
\begin{quote}
\begin{verbatim}
$WL_HOME/myserver/servletclasses/opencms/ExternalComponents/xerces.jar
$WL_HOME/myserver/servletclasses/opencms/logs/
$WL_HOME/myserver/servletclasses/opencms/opencms.jar
$WL_HOME/myserver/servletclasses/opencms/opencms.properties
\end{verbatim}
\end{quote}


\item \textbf{Step 5:} In BEA WebLogic with Oracle, start \texttt{sqlplus} and source in
the Oracle setup script, e.g.:
\begin{quote}
\texttt{sqlplus\\}
\textit{ ... <login to DB etc.>} \texttt{\\
@oraclePlsql.sql \\
quit}
\end{quote}

Be aware that this erases all pre-existent OpenCms data of previous
OpenCms installations.

\item \textbf{Step 6:} In BEA WebLogic, I use the log file \\
\texttt{\$WL\_HOME/myserver/servletclasses/opencms/logs/opencms.log}

Use the document root instead of the \texttt{(your apache docroot)} as described above, i.e.:

\begin{quote}
\begin{verbatim}
registry = (your opencms folder)/config/registry.xml
exportpoint.path.0 = $WL_HOME/myserver/public_html/pics/
exportpoint.path.1 = $WL_HOME/myserver/public_html/download/
exportpoint.path.2 = $WL_HOME/myserver/public_html/pics/system/
\end{verbatim}
\end{quote}

I have a special Solaris user and
group weblogic which owns all BEA WebLogic files (i.e., all files
including and below \texttt{\$WL\_HOME}).

\item \textbf{Step 7:} Add following lines to \texttt{\$WL\_HOME/weblogic.properties}
(replace \texttt{(path to \\
servletclasses)} by the complete path of your \texttt{servletclasses}
subfolder in your local \texttt{\$WL\_HOME} installation directory):

\begin{quote}
\begin{verbatim}
weblogic.httpd.register.opencms=com.opencms.core.OpenCmsServlet
weblogic.httpd.initArgs.opencms=properties= 
/(path to servletclasses)/opencms/opencms.properties
\end{verbatim}
\end{quote}


Change following lines in \texttt{\$WL\_HOME/startWebLogic.sh} from:

\begin{quote}
\begin{verbatim}
PRE_CLASSPATH=
POST_CLASSPATH=
\end{verbatim}
\end{quote}

to:

\begin{quote}
\begin{verbatim}
PRE_CLASSPATH="${WL_HOME}/myserver/servletclasses/opencms/opencms.jar:
${WL_HOME}/lib/xerces.jar:${J2EE_HOME}/lib/j2ee.jar"

POST_CLASSPATH="${ORACLE_HOME}/jdbc/lib/classes12.zip:
${ORACLE_HOME}/jdbc/lib/nls_charset12_01.zip"
\end{verbatim}
\end{quote}

This assumes that \texttt{\$ORACLE\_HOME} points towards your Oracle installation
directory, since the Oracle driver classes are loaded from
there. \texttt{\$J2EE} points towards your J2EE installation. I've put the
\texttt{xerces.jar} into \texttt{\$WL\_HOME/lib} instead into the \texttt{ExternalComponents}
directory.

\item \textbf{Step 8:} For BEA WebLogic, use this URL instead: \\
http://your.server:7001/opencms/system/workplace/action/login.html.

\item \textbf{Step 9:} BEA WebLogic puts its logging entries into
\texttt{\$WL\_HOME/myserver/weblogic.log}

\end{itemize}

\section{Browser settings}

\index{Browser settings}
\index{Configuring the Browser}
\index{Internet Explorer}
\index{Netscape}Browser settings
\index{ActiveX controls}
During the development of OpenCms, we focused on the implementation of the server components. 
To save time during the development we decided to use existing components for specific client 
functions. This means that ActiceX components were used for the user editors. 
The controls (there are 2, one for the WYSIWYG editor and one for the source code editor) 
are freely available, i.e. are shareware. This, however, currently restricts the WYSIWYG 
editor to Microsoft's Internet Explorer (Version 4 or higher). 
A thin version of Netscape Navigator is only available 
as source code editor. It would be possible to replace the ActiveX controls with Java applets. 
However, the OpenCms Team is currently not aware of an open source applet that could be 
plugged in. 
Should anyone know of an open source applet or have even  written one that can be used as a 
WYSIWYG  or source text editor,  we would, of course, be more than happy to hear about it.

\subsection{Configuring Microsoft Internet Explorer} 

On the menu bar, click on the menu item "View" and select "Internet options" 
(Tools - Internet Options for IE5). Then do the following: 
\begin{itemize}
\item On the tab "Security", select "Trusted site zones" from the drop-down menu and click on 
  "Add Sites" to add the URL (e.g. http://opencms.mycompany.com - ask your system administrator 
  for the exact URL) of the zone's OpenCms server. Deactivate the radio button "Require server 
  verification (https:) for all sites in this zone."

\item On the tab "Security", select "Trusted site zones" from the drop-down menu and click on 
  "Settings". All ActiveX control elements must be set to "Enable." A note on security: 
  It is safe to use ActiveX  controls with these settings because the controls are executed 
  by the OpenCms server only, and remain disabled for all other web sites.

\item On the tab "General" go to the section "Temporary Internet Files" and click on "Settings." 
  Select the option "Every visit to the page" (Check for a newer version of stored pages).
\end{itemize}

Cookies must be enabled.
 

\subsection{Installing ActiveX controls}

\begin{itemize}
\item The following ActiveX controls are used: \\
  Microsoft Dynamic HTML Editing Component and AY Software LeEdit OCX Control \\
\item Please read the respective help files when you install the ActiveX controls.\\
\item You can get LeEdit at \\
  \rqhttp{http://www.aysoft.com/ledit/share.htm}{http://www.aysoft.com/ledit/share.htm}\\ 
  and the other one at \\ 
  \rqhttp{http://msdn.microsoft.com/workshop/author/dhtml/edit/default.asp}{http://msdn.microsoft.com/workshop/author/dhtml/edit/default.asp}\\
  (the last one is aleady included in IE 5 or higher).
\end{itemize}


\subsection{Configuring Netscape Navigator}

On the menu bar, click on the menu item "Edit" and select "Settings." 
Then do the following: 
\begin{itemize}
\item On the tab "Settings" go to the section "Extended" and enable the options "Java" and "Java 
  Script." 
\item On the tab "Settings" go to the section "Extended/Cache" and enable "Compare cache and 
  network document: always." 
\end{itemize}
Cookies must be enabled.