\begin{PRE}
All Packages  Class Hierarchy  This Package  Previous  Next  Index
\end{PRE}

\htmlHR

\section*{  Class com.opencms.file.CmsFile }

\begin{PRE}
com.opencms.file.CmsResource
   {\htmlBar}
   +----com.opencms.file.CmsFile
\end{PRE}

\htmlHR

\begin{description}
\item public class {\bf CmsFile}  
\item extends CmsResource 
\end{description}

This class describes a file in the Cms. 

\begin{description}
\item {\bf Version:}  

\$Revision: 1.3 $ \$Date: 2002/03/07 15:58:17 $  
\item {\bf Author:}  

Michael Emmerich 
\end{description}

\htmlHR

\subsection*{  Variable Index }

\begin{description}
\item o {\bf m\_fileContent}  

The content of the file. 
\end{description}

\subsection*{  Constructor Index }

\begin{description}
\item o {\bf CmsFile}(int, int, int, String, int, int, int, int, int, int,
int, int, int, String, long, long, int, byte[], int, int)  

Constructor, creates a new CmsFile object. 
\end{description}

\subsection*{  Method Index }

\begin{description}
\item o {\bf clone}()  

Clones the CmsFile by creating a new CmsFolder.  
\item o {\bf getContents}()  

Gets the content of this file.  
\item o {\bf getExtension}()  

Gets the file-extension.  
\item o {\bf setContents}(byte[])  

Sets the content of this file. 
\end{description}

\subsection*{  Variables }

o {\bf m\_fileContent} 

\begin{PRE}
 private byte m\_fileContent[]
\end{PRE}

\begin{description}
\htmlDD The content of the file.

\end{description}

\subsection*{  Constructors }

o {\bf CmsFile} 

\begin{PRE}
 public CmsFile(int resourceId,
                int parentId,
                int fileId,
                String resourceName,
                int resourceType,
                int resourceFlags,
                int user,
                int group,
                int projectId,
                int accessFlags,
                int state,
                int lockedBy,
                int launcherType,
                String launcherClassname,
                long dateCreated,
                long dateLastModified,
                int resourceLastModifiedBy,
                byte fileContent[],
                int size,
                int lockedInProject)
\end{PRE}

\begin{description}
\htmlDD Constructor, creates a new CmsFile object. 

\begin{description}
\item {\bf Parameters:}  

resourceId - The database Id.  

parentId - The database Id of the parent folder.  

fileId - The id of the content.  

resourceName - The name (including complete path) of the resouce.  

resourceType - The type of this resource.  

rescourceFlags - The flags of thei resource.  

userId - The id of the user of this resource.  

groupId - The id of the group of this resource.  

projectId - The project id this resource belongs to.  

accessFlags - The access flags of this resource.  

state - The state of this resource.  

lockedBy - The user id of the user who has locked this resource.  

launcherType - The launcher that is require to process this recource.  

launcherClassname - The name of the Java class invoked by the launcher.  

dateCreated - The creation date of this resource.  

dateLastModified - The date of the last modification of the resource.  

fileContent - Then content of the file.  

resourceLastModifiedBy - The user who changed the file.  

size - The size of the file content.  
\end{description}

\end{description}

\subsection*{  Methods }

o {\bf clone} 

\begin{PRE}
 public Object clone()
\end{PRE}

\begin{description}
\htmlDD Clones the CmsFile by creating a new CmsFolder. 

\begin{description}
\item {\bf Returns:}  

Cloned CmsFile.  
\item {\bf Overrides:}  

clone in class CmsResource  
\end{description}

\end{description}

o {\bf getContents} 

\begin{PRE}
 public byte[] getContents()
\end{PRE}

\begin{description}
\htmlDD Gets the content of this file. 

\begin{description}
\item {\bf Returns:}  

the content of this file.  
\end{description}

\end{description}

o {\bf getExtension} 

\begin{PRE}
 public String getExtension()
\end{PRE}

\begin{description}
\htmlDD Gets the file-extension. 

\begin{description}
\item {\bf Returns:}  

the file extension. If this file has no extension, it returns a empty string
(``'').  
\end{description}

\end{description}

o {\bf setContents} 

\begin{PRE}
 public void setContents(byte value[])
\end{PRE}

\begin{description}
\htmlDD Sets the content of this file. 

\begin{description}
\item {\bf Parameters:}  

value - the content of this file.  
\end{description}

\end{description}

\htmlHR

\begin{PRE}
All Packages  Class Hierarchy  This Package  Previous  Next  Index
\end{PRE}

