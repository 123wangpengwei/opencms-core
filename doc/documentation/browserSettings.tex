\section{Client Setup}

\textbf{All configurations issues on this page only apply when you
set up a workstation to access the backoffice part of OpenCms.}
This is the part where you can edit pages, create new pages,
manage users etc. In OpenCms this is called "the Workplace" and it
makes use of dynamic HTML and some plugins to provide editing
functionality that can not be achieved using HTML alone.

\textbf{The HTML generated by OpenCms as output for the final
website is 100\% controlled by you and no extra client setup or
plugin is needed.}

For the development of OpenCms, we so far are focused on the
implementation of the server components. To save time during the
development we decided to use existing components for specific
client functions. This means that ActiveX components were used for
the user editors to offer the end user a rich set of
functionality. This, however, currently restricts the WYSIWYG
editor to Microsoft Internet Explorer. For Netscape Navigator, a
simple Textarea is used as source code editor for both WYSIWYG and
source code mode.

It would be possible to replace the ActiveX controls with Java
applets. However, the OpenCms Group is currently not aware of an
open source applet that could be plugged in. Should anyone know of
an open source applet or have even written one that can be used as
a WYSIWYG or advanced source code editor, we would, of course, be
more than happy to use it in OpenCms.


\subsection{Configuring MS Internet Explorer 5.x and 6.x Clients}

First step is the installation of the neccessary controls. There are
two ActiveX controls needed by OpenCms:

\begin{enumerate}
\item For the WYSIWYG editor, the "Dynamic HTML Edit Control" is
used. This control is part of all MS IE installations since
version 5.0, which means you must make sure the IE you use is
version 5.0 or later.
\item The code editor is a component developed by 
    \rqhttp{http://www.aysoft.com}{AY Software} and it's
called LeEdit OCX Control. You can download the shareware version
of this control from the site:\\
\rqhttp{http://www.aysoft.com/ledit.htm}{http://www.aysoft.com/ledit.htm}.
This control must be installed on all clients that need access to
the code editor functionality.
\end{enumerate}

The second step is configuring the ActiveX settings so that the
controls work properly. Open IEs "Internet options." Then do the
following:

\begin{enumerate}
\item On the tab "Security", select "Trusted site zones" from the
drop-down menu and click on "Add Sites" to add the URL (e.g.
http://opencms.mycompany.com - ask your system administrator for
the exact URL) of the zone's OpenCms server. Deactivate the radio
button "Require server verification (https:) for all sites in this
zone."
\item On the tab "Security", select "Trusted site zones"
from the drop-down menu and click on "Settings". All ActiveX
control elements must be set to "Enable." A note on security: It
is safe to use ActiveX controls with these settings since their
use is allowed only for the "Trusted sites", and ActiveX remains
disabled for all other web sites.
\end{enumerate}

This setup must be repeated for all clients / workstations that
use the OpenCms workplace. Cookies must be enabled on this
machines.


\subsection{Configuring Netscape Navigator 4.x Clients}

Ok, honestly we try our best to keep the workplace compatible with
Netscape versions 4.x. We so far do not support Netscape 6.0 or
other browsers that use the new Mozilla HTML rendering engine.
That's a pity since Mozilla is open source as well. If anyone of
you feels compelled to help boosting Netscape compatibility for
4.x or 6.0 (at last for the HTML parts of the workplace), please
send an Email to
\rqhttp{mailto://contributions@opencms.org}{contributions@opencms.org}.

Settings for Netscape 4.x:
\begin{enumerate}
\item On the menu bar, click on the menu item "Edit" and select
"Settings." Then do the following:
\item On the tab "Settings" go
to the section "Extended" and enable the options "Java" and "Java
Script."
\item On the tab "Settings" go to the section
"Extended/Cache" and enable "Compare cache and network document:
always."
\end{enumerate}

This setup must be repeated for all clients / workstations that
use the OpenCms workplace. Cookies must be enabled on this
machines.
