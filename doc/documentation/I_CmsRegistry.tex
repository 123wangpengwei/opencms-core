\begin{PRE}
All Packages  Class Hierarchy  This Package  Previous  Next  Index
\end{PRE}

\htmlHR

\section*{  Interface com.opencms.file.I\_CmsRegistry }

\begin{description}
\item public interface {\bf I\_CmsRegistry} 
\end{description}

This interface describes the registry for OpenCms. 

\begin{description}
\item {\bf Version:}  

\$Revision: 1.3 $ \$Date: 2002/03/07 15:58:17 $  
\item {\bf Author:}  

Andreas Schouten 
\end{description}

\htmlHR

\subsection*{  Variable Index }

\begin{description}
\item o {\bf C\_ANY\_VERSION}  

\item o {\bf C\_MODULE\_PATH}  

The name of the folder to extend the exportpath 
\end{description}

\subsection*{  Method Index }

\begin{description}
\item o {\bf clone}(CmsObject)  

This method clones the registry.  
\item o {\bf createModule}(String, String, String, String, long, int)  

This method creates a new module in the repository.  
\item o {\bf createModule}(String, String, String, String, String, int)  

This method creates a new module in the repository.  
\item o {\bf deleteCheckDependencies}(String)  

This method checks which modules need this module.  
\item o {\bf deleteGetConflictingFileNames}(String, Vector, Vector, Vector,
Vector, Vector)  

This method checks for conflicting files before the deletion of a module.  
\item o {\bf deleteModule}(String, Vector)  

Deletes a module.  
\item o {\bf deleteModuleView}(String)  

Deletes the view for a module.  
\item o {\bf exportModule}(String, String[], String)  

This method exports a module to the filesystem.  
\item o {\bf getModuleAuthor}(String)  

This method returns the author of the module.  
\item o {\bf getModuleAuthorEmail}(String)  

This method returns the email of author of the module.  
\item o {\bf getModuleCreateDate}(String)  

Gets the create date of the module.  
\item o {\bf getModuleDependencies}(String, Vector, Vector, Vector)  

Returns the module dependencies for the module.  
\item o {\bf getModuleDescription}(String)  

Returns the description of the module.  
\item o {\bf getModuleDocumentPath}(String)  

Gets the url to the documentation of the module.  
\item o {\bf getModuleExportables}(Hashtable)  

Returns all exportable classes for all modules.  
\item o {\bf getModuleFiles}(String, Vector, Vector)  

Returns all filenames and hashcodes belonging to the module.  
\item o {\bf getModuleMaintenanceEventClass}(String)  

Returns the class, that receives all maintenance-events for the module.  
\item o {\bf getModuleMaintenanceEventName}(String)  

Returns the name of the class, that receives all maintenance-events for the
module.  
\item o {\bf getModuleNames}()  

Returns the names of all available modules.  
\item o {\bf getModuleNiceName}(String)  

Gets the nice name of the module.  
\item o {\bf getModuleParameter}(String, String)  

Gets a parameter for a module.  
\item o {\bf getModuleParameterBoolean}(String, String)  

Returns a parameter for a module.  
\item o {\bf getModuleParameterBoolean}(String, String, Boolean)  

Returns a parameter for a module.  
\item o {\bf getModuleParameterBoolean}(String, String, boolean)  

Returns a parameter for a module.  
\item o {\bf getModuleParameterByte}(String, String)  

Returns a parameter for a module.  
\item o {\bf getModuleParameterByte}(String, String, Byte)  

Returns a parameter for a module.  
\item o {\bf getModuleParameterByte}(String, String, byte)  

Returns a parameter for a module.  
\item o {\bf getModuleParameterDescription}(String, String)  

Returns a description for parameter in a module.  
\item o {\bf getModuleParameterDouble}(String, String)  

Returns a parameter for a module.  
\item o {\bf getModuleParameterDouble}(String, String, double)  

Returns a parameter for a module.  
\item o {\bf getModuleParameterDouble}(String, String, Double)  

Returns a parameter for a module.  
\item o {\bf getModuleParameterFloat}(String, String)  

Returns a parameter for a module.  
\item o {\bf getModuleParameterFloat}(String, String, Float)  

Returns a parameter for a module.  
\item o {\bf getModuleParameterFloat}(String, String, float)  

Returns a parameter for a module.  
\item o {\bf getModuleParameterInteger}(String, String)  

Returns a parameter for a module.  
\item o {\bf getModuleParameterInteger}(String, String, int)  

Returns a parameter for a module.  
\item o {\bf getModuleParameterInteger}(String, String, Integer)  

Returns a parameter for a module.  
\item o {\bf getModuleParameterLong}(String, String)  

Returns a parameter for a module.  
\item o {\bf getModuleParameterLong}(String, String, long)  

Returns a parameter for a module.  
\item o {\bf getModuleParameterLong}(String, String, Long)  

Returns a parameter for a module.  
\item o {\bf getModuleParameterNames}(String)  

Gets all parameter-names for a module.  
\item o {\bf getModuleParameterString}(String, String)  

Returns a parameter for a module.  
\item o {\bf getModuleParameterString}(String, String, String)  

Returns a parameter for a module.  
\item o {\bf getModuleParameterType}(String, String)  

This method returns the type of a parameter in a module.  
\item o {\bf getModulePublishables}(Vector, String)  

Returns all publishable classes for all modules.  
\item o {\bf getModuleRepositories}(String)  

Returns all repositories for a module.  
\item o {\bf getModuleUploadDate}(String)  

Returns the upload-date for the module.  
\item o {\bf getModuleUploadedBy}(String)  

Returns the user-name of the user who had uploaded the module.  
\item o {\bf getModuleVersion}(String)  

This method returns the version of the module.  
\item o {\bf getModuleViewName}(String)  

Returns the name of the view, that is implemented by the module.  
\item o {\bf getModuleViewUrl}(String)  

Returns the url to the view-url for the module within the system.  
\item o {\bf getRepositories}()  

Returns all repositories for all modules.  
\item o {\bf getResourceTypes}(Vector, Vector, Vector, Vector)  

Returns all Resourcetypes and korresponding parameter for System and all
modules.  
\item o {\bf getSystemValue}(String)  

Returns a value for a system-key.  
\item o {\bf getSystemValues}(String)  

Returns a vector of value for a system-key.  
\item o {\bf getViews}(Vector, Vector)  

Returns all views and korresponding urls for all modules.  
\item o {\bf importCheckDependencies}(String)  

Checks the dependencies for a new Module.  
\item o {\bf importGetConflictingFileNames}(String)  

Checks for files that already exist in the system but should be replaced by
the module.  
\item o {\bf importGetModuleName}(String)  

Returns the name of the module to be imported.  
\item o {\bf importGetResourcesForProject}(String)  

Returns all files that are needed to create a project for the module-import.  
\item o {\bf importModule}(String, Vector)  

Imports a module.  
\item o {\bf moduleExists}(String)  

Checks if the module exists already in the repository.  
\item o {\bf setModuleAuthor}(String, String)  

This method sets the author of the module.  
\item o {\bf setModuleAuthorEmail}(String, String)  

This method sets the email of author of the module.  
\item o {\bf setModuleCreateDate}(String, long)  

Sets the create date of the module.  
\item o {\bf setModuleCreateDate}(String, String)  

Sets the create date of the module.  
\item o {\bf setModuleDependencies}(String, Vector, Vector, Vector)  

Sets the module dependencies for the module.  
\item o {\bf setModuleDescription}(String, String)  

Sets the description of the module.  
\item o {\bf setModuleDocumentPath}(String, String)  

Sets the url to the documentation of the module.  
\item o {\bf setModuleMaintenanceEventClass}(String, String)  

Sets the classname, that receives all maintenance-events for the module.  
\item o {\bf setModuleNiceName}(String, String)  

Sets the description of the module.  
\item o {\bf setModuleParameter}(String, String, boolean)  

Sets a parameter for a module.  
\item o {\bf setModuleParameter}(String, String, Boolean)  

Sets a parameter for a module.  
\item o {\bf setModuleParameter}(String, String, Byte)  

Sets a parameter for a module.  
\item o {\bf setModuleParameter}(String, String, byte)  

Sets a parameter for a module.  
\item o {\bf setModuleParameter}(String, String, Double)  

Sets a parameter for a module.  
\item o {\bf setModuleParameter}(String, String, double)  

Sets a parameter for a module.  
\item o {\bf setModuleParameter}(String, String, Float)  

Sets a parameter for a module.  
\item o {\bf setModuleParameter}(String, String, float)  

Sets a parameter for a module.  
\item o {\bf setModuleParameter}(String, String, int)  

Sets a parameter for a module.  
\item o {\bf setModuleParameter}(String, String, Integer)  

Sets a parameter for a module.  
\item o {\bf setModuleParameter}(String, String, Long)  

Sets a parameter for a module.  
\item o {\bf setModuleParameter}(String, String, long)  

Sets a parameter for a module.  
\item o {\bf setModuleParameter}(String, String, String)  

Sets a parameter for a module.  
\item o {\bf setModuleParameterdef}(String, Vector, Vector, Vector, Vector)  

Sets the module dependencies for the module.  
\item o {\bf setModuleRepositories}(String, String[])  

Sets all repositories for a module.  
\item o {\bf setModuleVersion}(String, int)  

This method sets the version of the module.  
\item o {\bf setModuleView}(String, String, String)  

Sets a view for a module  
\item o {\bf setSystemValue}(String, String)  

Public method to set system values.  
\item o {\bf setSystemValues}(String, Hashtable)  

Public method to set system values with hashtable. 
\end{description}

\subsection*{  Variables }

o {\bf C\_ANY\_VERSION} 

\begin{PRE}
 public static final int C\_ANY\_VERSION
\end{PRE}

o {\bf C\_MODULE\_PATH} 

\begin{PRE}
 public static final String C\_MODULE\_PATH
\end{PRE}

\begin{description}
\htmlDD The name of the folder to extend the exportpath

\end{description}

\subsection*{  Methods }

o {\bf clone} 

\begin{PRE}
 public abstract I\_CmsRegistry clone(CmsObject cms)
\end{PRE}

\begin{description}
\htmlDD This method clones the registry. 

\begin{description}
\item {\bf Parameters:}  

CmsObject - the current cms-object for the user.  
\item {\bf Returns:}  

the cloned registry.  
\end{description}

\end{description}

o {\bf createModule} 

\begin{PRE}
 public abstract void createModule(String modulename,
                                   String niceModulename,
                                   String description,
                                   String author,
                                   long createDate,
                                   int version) throws CmsException
\end{PRE}

\begin{description}
\htmlDD This method creates a new module in the repository. 

\begin{description}
\item {\bf Parameters:}  

String - modulename the name of the module.  

String - niceModulename another name of the module.  

String - description the description of the module.  

String - author the name of the author.  

long - createDate the creation date of the module  

int - version the version number of the module.  
\end{description}

\end{description}

o {\bf createModule} 

\begin{PRE}
 public abstract void createModule(String modulename,
                                   String niceModulename,
                                   String description,
                                   String author,
                                   String createDate,
                                   int version) throws CmsException
\end{PRE}

\begin{description}
\htmlDD This method creates a new module in the repository. 

\begin{description}
\item {\bf Parameters:}  

String - modulename the name of the module.  

String - niceModulename another name of the module.  

String - description the description of the module.  

String - author the name of the author.  

String - createDate the creation date of the module in the format: mm.dd.yyyy 


int - version the version number of the module.  
\end{description}

\end{description}

o {\bf deleteCheckDependencies} 

\begin{PRE}
 public abstract Vector deleteCheckDependencies(String modulename) throws CmsException
\end{PRE}

\begin{description}
\htmlDD This method checks which modules need this module. If a module depends
on this the name will be returned in the vector. 

\begin{description}
\item {\bf Parameters:}  

modulename - The name of the module to check.  
\item {\bf Returns:}  

s a Vector with modulenames that depends on the overgiven module.  
\end{description}

\end{description}

o {\bf deleteGetConflictingFileNames} 

\begin{PRE}
 public abstract void deleteGetConflictingFileNames(String modulename,
                                                    Vector filesWithProperty,
                                                    Vector missingFiles,
                                                    Vector wrongChecksum,
                                                    Vector filesInUse,
                                                    Vector resourcesForProject) throws CmsException
\end{PRE}

\begin{description}
\htmlDD This method checks for conflicting files before the deletion of a
module. It uses several Vectors to return the different conflicting files. 

\begin{description}
\item {\bf Parameters:}  

modulename - the name of the module that should be deleted.  

filesWithProperty - a return value. The files that are marked with the
module-property for this module.  

missingFiles - a return value. The files that are missing.  

wrongChecksum - a return value. The files that should be deleted but have
another checksum as at import-time.  

filesInUse - a return value. The files that should be deleted but are in use
by other modules.  

resourcesForProject - a return value. The files that should be copied to a
project to delete.  
\end{description}

\end{description}

o {\bf deleteModule} 

\begin{PRE}
 public abstract void deleteModule(String module,
                                   Vector exclusion) throws CmsException
\end{PRE}

\begin{description}
\htmlDD Deletes a module. This method is synchronized, so only one module can
be deleted at one time. 

\begin{description}
\item {\bf Parameters:}  

module-name - the name of the module that should be deleted.  

exclusion - a Vector with resource-names that should be excluded from this
deletion.  
\end{description}

\end{description}

o {\bf deleteModuleView} 

\begin{PRE}
 public abstract void deleteModuleView(String modulename) throws CmsException
\end{PRE}

\begin{description}
\htmlDD Deletes the view for a module. 

\begin{description}
\item {\bf Parameters:}  

String - the name of the module.  
\end{description}

\end{description}

o {\bf exportModule} 

\begin{PRE}
 public abstract void exportModule(String moduleName,
                                   String resources[],
                                   String fileName) throws CmsException
\end{PRE}

\begin{description}
\htmlDD This method exports a module to the filesystem. 

\begin{description}
\item {\bf Parameters:}  

moduleName - the name of the module to be exported.  

String[] - an array of resources to be exported.  

fileName - the name of the file to write the export to.  
\end{description}

\end{description}

o {\bf getModuleAuthor} 

\begin{PRE}
 public abstract String getModuleAuthor(String modulename)
\end{PRE}

\begin{description}
\htmlDD This method returns the author of the module. 

\begin{description}
\item {\bf Parameters:}  

eter - String the name of the module.  
\item {\bf Returns:}  

java.lang.String the author of the module.  
\end{description}

\end{description}

o {\bf getModuleAuthorEmail} 

\begin{PRE}
 public abstract String getModuleAuthorEmail(String modulename)
\end{PRE}

\begin{description}
\htmlDD This method returns the email of author of the module. 

\begin{description}
\item {\bf Parameters:}  

eter - String the name of the module.  
\item {\bf Returns:}  

java.lang.String the email of author of the module.  
\end{description}

\end{description}

o {\bf getModuleCreateDate} 

\begin{PRE}
 public abstract long getModuleCreateDate(String modulname)
\end{PRE}

\begin{description}
\htmlDD Gets the create date of the module. 

\begin{description}
\item {\bf Parameters:}  

eter - String the name of the module.  
\item {\bf Returns:}  

long the create date of the module.  
\end{description}

\end{description}

o {\bf getModuleDependencies} 

\begin{PRE}
 public abstract int getModuleDependencies(String modulename,
                                           Vector modules,
                                           Vector minVersions,
                                           Vector maxVersions)
\end{PRE}

\begin{description}
\htmlDD Returns the module dependencies for the module. 

\begin{description}
\item {\bf Parameters:}  

module - String the name of the module to check.  

modules - Vector in this parameter the names of the dependend modules will be
returned.  

minVersions - Vector in this parameter the minimum versions of the dependend
modules will be returned.  

maxVersions - Vector in this parameter the maximum versions of the dependend
modules will be returned.  
\item {\bf Returns:}  

int the amount of dependencies for the module will be returned.  
\end{description}

\end{description}

o {\bf getModuleDescription} 

\begin{PRE}
 public abstract String getModuleDescription(String module)
\end{PRE}

\begin{description}
\htmlDD Returns the description of the module. 

\begin{description}
\item {\bf Parameters:}  

eter - String the name of the module.  
\item {\bf Returns:}  

java.lang.String the description of the module.  
\end{description}

\end{description}

o {\bf getModuleDocumentPath} 

\begin{PRE}
 public abstract String getModuleDocumentPath(String modulename)
\end{PRE}

\begin{description}
\htmlDD Gets the url to the documentation of the module. 

\begin{description}
\item {\bf Parameters:}  

eter - String the name of the module.  
\item {\bf Returns:}  

java.lang.String the url to the documentation of the module.  
\end{description}

\end{description}

o {\bf getModuleFiles} 

\begin{PRE}
 public abstract int getModuleFiles(String modulename,
                                    Vector retNames,
                                    Vector retCodes)
\end{PRE}

\begin{description}
\htmlDD Returns all filenames and hashcodes belonging to the module. 

\begin{description}
\item {\bf Parameters:}  

String - modulname the name of the module.  

retNames - the names of the resources belonging to the module.  

retCodes - the hashcodes of the resources belonging to the module.  
\item {\bf Returns:}  

the amount of entrys.  
\end{description}

\end{description}

o {\bf getModuleMaintenanceEventClass} 

\begin{PRE}
 public abstract Class getModuleMaintenanceEventClass(String modulname)
\end{PRE}

\begin{description}
\htmlDD Returns the class, that receives all maintenance-events for the
module. 

\begin{description}
\item {\bf Parameters:}  

eter - String the name of the module.  
\item {\bf Returns:}  

java.lang.Class that receives all maintenance-events for the module.  
\end{description}

\end{description}

o {\bf getModuleMaintenanceEventName} 

\begin{PRE}
 public abstract String getModuleMaintenanceEventName(String modulname)
\end{PRE}

\begin{description}
\htmlDD Returns the name of the class, that receives all maintenance-events
for the module. 

\begin{description}
\item {\bf Parameters:}  

eter - String the name of the module.  
\item {\bf Returns:}  

java.lang.Class that receives all maintenance-events for the module.  
\end{description}

\end{description}

o {\bf getModuleNames} 

\begin{PRE}
 public abstract Enumeration getModuleNames()
\end{PRE}

\begin{description}
\htmlDD Returns the names of all available modules. 

\begin{description}
\item {\bf Returns:}  

String[] the names of all available modules.  
\end{description}

\end{description}

o {\bf getModuleNiceName} 

\begin{PRE}
 public abstract String getModuleNiceName(String module)
\end{PRE}

\begin{description}
\htmlDD Gets the nice name of the module. 

\begin{description}
\item {\bf Parameters:}  

String - the name of the module.  
\item {\bf Returns:}  

s String the nice name of the module.  
\end{description}

\end{description}

o {\bf getModuleParameter} 

\begin{PRE}
 public abstract String getModuleParameter(String modulename,
                                           String parameter)
\end{PRE}

\begin{description}
\htmlDD Gets a parameter for a module. 

\begin{description}
\item {\bf Parameters:}  

modulename - java.lang.String the name of the module.  

parameter - java.lang.String the name of the parameter to set.  
\item {\bf Returns:}  

value java.lang.String the value to set for the parameter.  
\end{description}

\end{description}

o {\bf getModuleParameterBoolean} 

\begin{PRE}
 public abstract boolean getModuleParameterBoolean(String modulname,
                                                   String parameter)
\end{PRE}

\begin{description}
\htmlDD Returns a parameter for a module. 

\begin{description}
\item {\bf Parameters:}  

modulname - String the name of the module.  

parameter - String the name of the parameter.  
\item {\bf Returns:}  

boolean the value for the parameter in the module.  
\end{description}

\end{description}

o {\bf getModuleParameterBoolean} 

\begin{PRE}
 public abstract Boolean getModuleParameterBoolean(String modulname,
                                                   String parameter,
                                                   Boolean defaultValue)
\end{PRE}

\begin{description}
\htmlDD Returns a parameter for a module. 

\begin{description}
\item {\bf Parameters:}  

modulname - String the name of the module.  

parameter - String the name of the parameter.  

default - the default value.  
\item {\bf Returns:}  

boolean the value for the parameter in the module.  
\end{description}

\end{description}

o {\bf getModuleParameterBoolean} 

\begin{PRE}
 public abstract boolean getModuleParameterBoolean(String modulname,
                                                   String parameter,
                                                   boolean defaultValue)
\end{PRE}

\begin{description}
\htmlDD Returns a parameter for a module. 

\begin{description}
\item {\bf Parameters:}  

modulname - String the name of the module.  

parameter - String the name of the parameter.  

default - the default value.  
\item {\bf Returns:}  

boolean the value for the parameter in the module.  
\end{description}

\end{description}

o {\bf getModuleParameterByte} 

\begin{PRE}
 public abstract byte getModuleParameterByte(String modulname,
                                             String parameter)
\end{PRE}

\begin{description}
\htmlDD Returns a parameter for a module. 

\begin{description}
\item {\bf Parameters:}  

modulname - String the name of the module.  

parameter - String the name of the parameter.  

default - the default value.  
\item {\bf Returns:}  

boolean the value for the parameter in the module.  
\end{description}

\end{description}

o {\bf getModuleParameterByte} 

\begin{PRE}
 public abstract byte getModuleParameterByte(String modulname,
                                             String parameter,
                                             byte defaultValue)
\end{PRE}

\begin{description}
\htmlDD Returns a parameter for a module. 

\begin{description}
\item {\bf Parameters:}  

modulname - String the name of the module.  

parameter - String the name of the parameter.  

default - the default value.  
\item {\bf Returns:}  

boolean the value for the parameter in the module.  
\end{description}

\end{description}

o {\bf getModuleParameterByte} 

\begin{PRE}
 public abstract Byte getModuleParameterByte(String modulname,
                                             String parameter,
                                             Byte defaultValue)
\end{PRE}

\begin{description}
\htmlDD Returns a parameter for a module. 

\begin{description}
\item {\bf Parameters:}  

modulname - String the name of the module.  

parameter - String the name of the parameter.  

default - the default value.  
\item {\bf Returns:}  

boolean the value for the parameter in the module.  
\end{description}

\end{description}

o {\bf getModuleParameterDescription} 

\begin{PRE}
 public abstract String getModuleParameterDescription(String modulname,
                                                      String parameter)
\end{PRE}

\begin{description}
\htmlDD Returns a description for parameter in a module. 

\begin{description}
\item {\bf Parameters:}  

modulname - String the name of the module.  

parameter - String the name of the parameter.  
\item {\bf Returns:}  

String the description for the parameter in the module.  
\end{description}

\end{description}

o {\bf getModuleParameterDouble} 

\begin{PRE}
 public abstract double getModuleParameterDouble(String modulname,
                                                 String parameter)
\end{PRE}

\begin{description}
\htmlDD Returns a parameter for a module. 

\begin{description}
\item {\bf Parameters:}  

modulname - String the name of the module.  

parameter - String the name of the parameter.  
\item {\bf Returns:}  

boolean the value for the parameter in the module.  
\end{description}

\end{description}

o {\bf getModuleParameterDouble} 

\begin{PRE}
 public abstract double getModuleParameterDouble(String modulname,
                                                 String parameter,
                                                 double defaultValue)
\end{PRE}

\begin{description}
\htmlDD Returns a parameter for a module. 

\begin{description}
\item {\bf Parameters:}  

modulname - String the name of the module.  

parameter - String the name of the parameter.  

default - the default value.  
\item {\bf Returns:}  

boolean the value for the parameter in the module.  
\end{description}

\end{description}

o {\bf getModuleParameterDouble} 

\begin{PRE}
 public abstract Double getModuleParameterDouble(String modulname,
                                                 String parameter,
                                                 Double defaultValue)
\end{PRE}

\begin{description}
\htmlDD Returns a parameter for a module. 

\begin{description}
\item {\bf Parameters:}  

modulname - String the name of the module.  

parameter - String the name of the parameter.  

default - the default value.  
\item {\bf Returns:}  

boolean the value for the parameter in the module.  
\end{description}

\end{description}

o {\bf getModuleParameterFloat} 

\begin{PRE}
 public abstract float getModuleParameterFloat(String modulname,
                                               String parameter)
\end{PRE}

\begin{description}
\htmlDD Returns a parameter for a module. 

\begin{description}
\item {\bf Parameters:}  

modulname - String the name of the module.  

parameter - String the name of the parameter.  

default - the default value.  
\item {\bf Returns:}  

boolean the value for the parameter in the module.  
\end{description}

\end{description}

o {\bf getModuleParameterFloat} 

\begin{PRE}
 public abstract float getModuleParameterFloat(String modulname,
                                               String parameter,
                                               float defaultValue)
\end{PRE}

\begin{description}
\htmlDD Returns a parameter for a module. 

\begin{description}
\item {\bf Parameters:}  

modulname - String the name of the module.  

parameter - String the name of the parameter.  

default - the default value.  
\item {\bf Returns:}  

boolean the value for the parameter in the module.  
\end{description}

\end{description}

o {\bf getModuleParameterFloat} 

\begin{PRE}
 public abstract Float getModuleParameterFloat(String modulname,
                                               String parameter,
                                               Float defaultValue)
\end{PRE}

\begin{description}
\htmlDD Returns a parameter for a module. 

\begin{description}
\item {\bf Parameters:}  

modulname - String the name of the module.  

parameter - String the name of the parameter.  

default - the default value.  
\item {\bf Returns:}  

boolean the value for the parameter in the module.  
\end{description}

\end{description}

o {\bf getModuleParameterInteger} 

\begin{PRE}
 public abstract int getModuleParameterInteger(String modulname,
                                               String parameter)
\end{PRE}

\begin{description}
\htmlDD Returns a parameter for a module. 

\begin{description}
\item {\bf Parameters:}  

modulname - String the name of the module.  

parameter - String the name of the parameter.  
\item {\bf Returns:}  

boolean the value for the parameter in the module.  
\end{description}

\end{description}

o {\bf getModuleParameterInteger} 

\begin{PRE}
 public abstract int getModuleParameterInteger(String modulname,
                                               String parameter,
                                               int defaultValue)
\end{PRE}

\begin{description}
\htmlDD Returns a parameter for a module. 

\begin{description}
\item {\bf Parameters:}  

modulname - String the name of the module.  

parameter - String the name of the parameter.  

default - the default value.  
\item {\bf Returns:}  

boolean the value for the parameter in the module.  
\end{description}

\end{description}

o {\bf getModuleParameterInteger} 

\begin{PRE}
 public abstract Integer getModuleParameterInteger(String modulname,
                                                   String parameter,
                                                   Integer defaultValue)
\end{PRE}

\begin{description}
\htmlDD Returns a parameter for a module. 

\begin{description}
\item {\bf Parameters:}  

modulname - String the name of the module.  

parameter - String the name of the parameter.  

default - the default value.  
\item {\bf Returns:}  

boolean the value for the parameter in the module.  
\end{description}

\end{description}

o {\bf getModuleParameterLong} 

\begin{PRE}
 public abstract long getModuleParameterLong(String modulname,
                                             String parameter)
\end{PRE}

\begin{description}
\htmlDD Returns a parameter for a module. 

\begin{description}
\item {\bf Parameters:}  

modulname - String the name of the module.  

parameter - String the name of the parameter.  

default - the default value.  
\item {\bf Returns:}  

boolean the value for the parameter in the module.  
\end{description}

\end{description}

o {\bf getModuleParameterLong} 

\begin{PRE}
 public abstract long getModuleParameterLong(String modulname,
                                             String parameter,
                                             long defaultValue)
\end{PRE}

\begin{description}
\htmlDD Returns a parameter for a module. 

\begin{description}
\item {\bf Parameters:}  

modulname - String the name of the module.  

parameter - String the name of the parameter.  

default - the default value.  
\item {\bf Returns:}  

boolean the value for the parameter in the module.  
\end{description}

\end{description}

o {\bf getModuleParameterLong} 

\begin{PRE}
 public abstract Long getModuleParameterLong(String modulname,
                                             String parameter,
                                             Long defaultValue)
\end{PRE}

\begin{description}
\htmlDD Returns a parameter for a module. 

\begin{description}
\item {\bf Parameters:}  

modulname - String the name of the module.  

parameter - String the name of the parameter.  

default - the default value.  
\item {\bf Returns:}  

boolean the value for the parameter in the module.  
\end{description}

\end{description}

o {\bf getModuleParameterNames} 

\begin{PRE}
 public abstract String[] getModuleParameterNames(String modulename)
\end{PRE}

\begin{description}
\htmlDD Gets all parameter-names for a module. 

\begin{description}
\item {\bf Parameters:}  

modulename - String the name of the module.  
\item {\bf Returns:}  

value String[] the names of the parameters for a module.  
\end{description}

\end{description}

o {\bf getModuleParameterString} 

\begin{PRE}
 public abstract String getModuleParameterString(String modulname,
                                                 String parameter)
\end{PRE}

\begin{description}
\htmlDD Returns a parameter for a module. 

\begin{description}
\item {\bf Parameters:}  

modulname - String the name of the module.  

parameter - String the name of the parameter.  
\item {\bf Returns:}  

boolean the value for the parameter in the module.  
\end{description}

\end{description}

o {\bf getModuleParameterString} 

\begin{PRE}
 public abstract String getModuleParameterString(String modulname,
                                                 String parameter,
                                                 String defaultValue)
\end{PRE}

\begin{description}
\htmlDD Returns a parameter for a module. 

\begin{description}
\item {\bf Parameters:}  

modulname - String the name of the module.  

parameter - String the name of the parameter.  

default - the default value.  
\item {\bf Returns:}  

boolean the value for the parameter in the module.  
\end{description}

\end{description}

o {\bf getModuleParameterType} 

\begin{PRE}
 public abstract String getModuleParameterType(String modulename,
                                               String parameter)
\end{PRE}

\begin{description}
\htmlDD This method returns the type of a parameter in a module. 

\begin{description}
\item {\bf Parameters:}  

modulename - the name of the module.  

parameter - the name of the parameter.  
\item {\bf Returns:}  

the type of the parameter.  
\end{description}

\end{description}

o {\bf getModuleRepositories} 

\begin{PRE}
 public abstract String[] getModuleRepositories(String modulename)
\end{PRE}

\begin{description}
\htmlDD Returns all repositories for a module. 

\begin{description}
\item {\bf Parameters:}  

eter - String modulname the name of the module.  
\item {\bf Returns:}  

java.lang.String[] the reprositories of a module.  
\end{description}

\end{description}

o {\bf getModuleUploadDate} 

\begin{PRE}
 public abstract long getModuleUploadDate(String modulename)
\end{PRE}

\begin{description}
\htmlDD Returns the upload-date for the module. 

\begin{description}
\item {\bf Parameters:}  

eter - String the name of the module.  
\item {\bf Returns:}  

java.lang.String the upload-date for the module.  
\end{description}

\end{description}

o {\bf getModuleUploadedBy} 

\begin{PRE}
 public abstract String getModuleUploadedBy(String module)
\end{PRE}

\begin{description}
\htmlDD Returns the user-name of the user who had uploaded the module. 

\begin{description}
\item {\bf Parameters:}  

eter - String the name of the module.  
\item {\bf Returns:}  

java.lang.String the user-name of the user who had uploaded the module.  
\end{description}

\end{description}

o {\bf getModuleVersion} 

\begin{PRE}
 public abstract int getModuleVersion(String modulename)
\end{PRE}

\begin{description}
\htmlDD This method returns the version of the module. 

\begin{description}
\item {\bf Parameters:}  

eter - String the name of the module.  
\item {\bf Returns:}  

java.lang.String the version of the module.  
\end{description}

\end{description}

o {\bf getModuleViewName} 

\begin{PRE}
 public abstract String getModuleViewName(String modulename)
\end{PRE}

\begin{description}
\htmlDD Returns the name of the view, that is implemented by the module. 

\begin{description}
\item {\bf Parameters:}  

eter - String the name of the module.  
\item {\bf Returns:}  

java.lang.String the name of the view, that is implemented by the module.  
\end{description}

\end{description}

o {\bf getModuleViewUrl} 

\begin{PRE}
 public abstract String getModuleViewUrl(String modulename)
\end{PRE}

\begin{description}
\htmlDD Returns the url to the view-url for the module within the system. 

\begin{description}
\item {\bf Parameters:}  

eter - String the name of the module.  
\item {\bf Returns:}  

java.lang.String the view-url to the module.  
\end{description}

\end{description}

o {\bf getModulePublishables} 

\begin{PRE}
 public abstract int getModulePublishables(Vector classes,
                                           String requiredMethod)
\end{PRE}

\begin{description}
\htmlDD Returns all publishable classes for all modules. 

\begin{description}
\item {\bf Parameters:}  

eter - Vector classes in this parameter the classess will be returned.  

eter - String requiredMethod The value of the methodTag for the different 
methods useable after publish.  null means the standard publish method 
``linkpublish'' means the method that needs the changed links as parameter
(i.e. search)  
\item {\bf Returns:}  

int the amount of classess.  
\end{description}

\end{description}

o {\bf getModuleExportables} 

\begin{PRE}
 public abstract int getModuleExportables(Hashtable classes)
\end{PRE}

\begin{description}
\htmlDD Returns all exportable classes for all modules. 

\begin{description}
\item {\bf Parameters:}  

eter - Hashtable classes in this parameter the classes will be returned.  
\item {\bf Returns:}  

int the amount of classes.  
\end{description}

\end{description}

o {\bf getRepositories} 

\begin{PRE}
 public abstract String[] getRepositories()
\end{PRE}

\begin{description}
\htmlDD Returns all repositories for all modules. 

\begin{description}
\item {\bf Returns:}  

java.lang.String[] the reprositories of all modules.  
\end{description}

\end{description}

o {\bf getResourceTypes} 

\begin{PRE}
 public abstract int getResourceTypes(Vector names,
                                      Vector launcherTypes,
                                      Vector launcherClass,
                                      Vector resourceClass)
\end{PRE}

\begin{description}
\htmlDD Returns all Resourcetypes and korresponding parameter for System and
all modules. 

\begin{description}
\item {\bf Parameters:}  

eter - Vector names in this parameter the names of the Resourcetypes will be
returned.  

eter - Vector launcherTypes in this parameters the launcherType will be
returned(int).  

eter - Vector launcherClass in this parameters the launcherClass will be
returned.  

eter - Vector resourceClass in this parameters the resourceClass will be
returned.  
\item {\bf Returns:}  

int the amount of resourcetypes.  
\end{description}

\end{description}

o {\bf getSystemValue} 

\begin{PRE}
 public abstract String getSystemValue(String key)
\end{PRE}

\begin{description}
\htmlDD Returns a value for a system-key. E.g. {\tt
{\htmlLt}system{\htmlGt}{\htmlLt}mailserver{\htmlGt}mail.server.com{\htmlLt}/m%
ailserver{\htmlGt}{\htmlLt}/system{\htmlGt}} can be requested via {\tt
getSystemValue(``mailserver'');} and returns ``mail.server.com. 

\begin{description}
\item {\bf Parameters:}  

eter - String the key of the system-value.  
\item {\bf Returns:}  

the value for that system-key.  
\end{description}

\end{description}

o {\bf getSystemValues} 

\begin{PRE}
 public abstract Hashtable getSystemValues(String key)
\end{PRE}

\begin{description}
\htmlDD Returns a vector of value for a system-key. 

\begin{description}
\item {\bf Parameters:}  

eter - String the key of the system-value.  
\item {\bf Returns:}  

the values for that system-key.  
\end{description}

\end{description}

o {\bf getViews} 

\begin{PRE}
 public abstract int getViews(Vector views,
                              Vector urls)
\end{PRE}

\begin{description}
\htmlDD Returns all views and korresponding urls for all modules. 

\begin{description}
\item {\bf Parameters:}  

eter - String[] views in this parameter the views will be returned.  

eter - String[] urls in this parameters the urls vor the views will be
returned.  
\item {\bf Returns:}  

int the amount of views.  
\end{description}

\end{description}

o {\bf importCheckDependencies} 

\begin{PRE}
 public abstract Vector importCheckDependencies(String moduleZip) throws CmsException
\end{PRE}

\begin{description}
\htmlDD Checks the dependencies for a new Module. 

\begin{description}
\item {\bf Parameters:}  

moduleZip - the name of the zipfile for the new module.  
\item {\bf Returns:}  

a Vector with dependencies that are not fullfilled.  
\end{description}

\end{description}

o {\bf importGetConflictingFileNames} 

\begin{PRE}
 public abstract Vector importGetConflictingFileNames(String moduleZip) throws CmsException
\end{PRE}

\begin{description}
\htmlDD Checks for files that already exist in the system but should be
replaced by the module. 

\begin{description}
\item {\bf Parameters:}  

moduleZip - The name of the zip-file to import.  
\item {\bf Returns:}  

s The complete paths to the resources that have conflicts.  
\end{description}

\end{description}

o {\bf importGetModuleName} 

\begin{PRE}
 public abstract String importGetModuleName(String moduleZip)
\end{PRE}

\begin{description}
\htmlDD Returns the name of the module to be imported. 

\begin{description}
\item {\bf Parameters:}  

moduleZip - the name of the zip-file to import from.  
\item {\bf Returns:}  

The name of the module to be imported.  
\end{description}

\end{description}

o {\bf importGetResourcesForProject} 

\begin{PRE}
 public abstract Vector importGetResourcesForProject(String moduleZip) throws CmsException
\end{PRE}

\begin{description}
\htmlDD Returns all files that are needed to create a project for the
module-import. 

\begin{description}
\item {\bf Parameters:}  

moduleZip - The name of the zip-file to import.  
\item {\bf Returns:}  

s The complete paths for resources that should be in the import-project.  
\end{description}

\end{description}

o {\bf importModule} 

\begin{PRE}
 public abstract void importModule(String moduleZip,
                                   Vector exclusion) throws CmsException
\end{PRE}

\begin{description}
\htmlDD Imports a module. This method is synchronized, so only one module can
be imported at on time. 

\begin{description}
\item {\bf Parameters:}  

moduleZip - the name of the zip-file to import from.  

exclusion - a Vector with resource-names that should be excluded from this
import.  
\end{description}

\end{description}

o {\bf moduleExists} 

\begin{PRE}
 public abstract boolean moduleExists(String modulename)
\end{PRE}

\begin{description}
\htmlDD Checks if the module exists already in the repository. 

\begin{description}
\item {\bf Parameters:}  

eter - String the name of the module.  
\item {\bf Returns:}  

true if the module exists, else false.  
\end{description}

\end{description}

o {\bf setModuleAuthor} 

\begin{PRE}
 public abstract void setModuleAuthor(String modulename,
                                      String author) throws CmsException
\end{PRE}

\begin{description}
\htmlDD This method sets the author of the module. 

\begin{description}
\item {\bf Parameters:}  

String - the name of the module.  

String - the name of the author.  
\end{description}

\end{description}

o {\bf setModuleAuthorEmail} 

\begin{PRE}
 public abstract void setModuleAuthorEmail(String modulename,
                                           String email) throws CmsException
\end{PRE}

\begin{description}
\htmlDD This method sets the email of author of the module. 

\begin{description}
\item {\bf Parameters:}  

String - the name of the module.  

String - the email of author of the module.  
\end{description}

\end{description}

o {\bf setModuleCreateDate} 

\begin{PRE}
 public abstract void setModuleCreateDate(String modulname,
                                          long createdate) throws CmsException
\end{PRE}

\begin{description}
\htmlDD Sets the create date of the module. 

\begin{description}
\item {\bf Parameters:}  

String - the name of the module.  

long - the create date of the module.  
\end{description}

\end{description}

o {\bf setModuleCreateDate} 

\begin{PRE}
 public abstract void setModuleCreateDate(String modulname,
                                          String createdate) throws CmsException
\end{PRE}

\begin{description}
\htmlDD Sets the create date of the module. 

\begin{description}
\item {\bf Parameters:}  

String - the name of the module.  

String - the create date of the module. Format: mm.dd.yyyy  
\end{description}

\end{description}

o {\bf setModuleDependencies} 

\begin{PRE}
 public abstract void setModuleDependencies(String modulename,
                                            Vector modules,
                                            Vector minVersions,
                                            Vector maxVersions) throws CmsException
\end{PRE}

\begin{description}
\htmlDD Sets the module dependencies for the module. 

\begin{description}
\item {\bf Parameters:}  

module - String the name of the module to check.  

modules - Vector in this parameter the names of the dependend modules will be
returned.  

minVersions - Vector in this parameter the minimum versions of the dependend
modules will be returned.  

maxVersions - Vector in this parameter the maximum versions of the dependend
modules will be returned.  
\end{description}

\end{description}

o {\bf setModuleDescription} 

\begin{PRE}
 public abstract void setModuleDescription(String module,
                                           String description) throws CmsException
\end{PRE}

\begin{description}
\htmlDD Sets the description of the module. 

\begin{description}
\item {\bf Parameters:}  

String - the name of the module.  

String - the description of the module.  
\end{description}

\end{description}

o {\bf setModuleDocumentPath} 

\begin{PRE}
 public abstract void setModuleDocumentPath(String modulename,
                                            String url) throws CmsException
\end{PRE}

\begin{description}
\htmlDD Sets the url to the documentation of the module. 

\begin{description}
\item {\bf Parameters:}  

String - the name of the module.  

java.lang.String - the url to the documentation of the module.  
\end{description}

\end{description}

o {\bf setModuleMaintenanceEventClass} 

\begin{PRE}
 public abstract void setModuleMaintenanceEventClass(String modulname,
                                                     String classname) throws CmsException
\end{PRE}

\begin{description}
\htmlDD Sets the classname, that receives all maintenance-events for the
module. 

\begin{description}
\item {\bf Parameters:}  

String - the name of the module.  

java.lang.Class - that receives all maintenance-events for the module.  
\end{description}

\end{description}

o {\bf setModuleNiceName} 

\begin{PRE}
 public abstract void setModuleNiceName(String module,
                                        String nicename) throws CmsException
\end{PRE}

\begin{description}
\htmlDD Sets the description of the module. 

\begin{description}
\item {\bf Parameters:}  

String - the name of the module.  

String - the nice name of the module.  
\end{description}

\end{description}

o {\bf setModuleParameter} 

\begin{PRE}
 public abstract void setModuleParameter(String modulename,
                                         String parameter,
                                         byte value) throws CmsException
\end{PRE}

\begin{description}
\htmlDD Sets a parameter for a module. 

\begin{description}
\item {\bf Parameters:}  

modulename - java.lang.String the name of the module.  

parameter - java.lang.String the name of the parameter to set.  

the - value to set for the parameter.  
\end{description}

\end{description}

o {\bf setModuleParameter} 

\begin{PRE}
 public abstract void setModuleParameter(String modulename,
                                         String parameter,
                                         double value) throws CmsException
\end{PRE}

\begin{description}
\htmlDD Sets a parameter for a module. 

\begin{description}
\item {\bf Parameters:}  

modulename - java.lang.String the name of the module.  

parameter - java.lang.String the name of the parameter to set.  

the - value to set for the parameter.  
\end{description}

\end{description}

o {\bf setModuleParameter} 

\begin{PRE}
 public abstract void setModuleParameter(String modulename,
                                         String parameter,
                                         float value) throws CmsException
\end{PRE}

\begin{description}
\htmlDD Sets a parameter for a module. 

\begin{description}
\item {\bf Parameters:}  

modulename - java.lang.String the name of the module.  

parameter - java.lang.String the name of the parameter to set.  

the - value to set for the parameter.  
\end{description}

\end{description}

o {\bf setModuleParameter} 

\begin{PRE}
 public abstract void setModuleParameter(String modulename,
                                         String parameter,
                                         int value) throws CmsException
\end{PRE}

\begin{description}
\htmlDD Sets a parameter for a module. 

\begin{description}
\item {\bf Parameters:}  

modulename - java.lang.String the name of the module.  

parameter - java.lang.String the name of the parameter to set.  

the - value to set for the parameter.  
\end{description}

\end{description}

o {\bf setModuleParameter} 

\begin{PRE}
 public abstract void setModuleParameter(String modulename,
                                         String parameter,
                                         long value) throws CmsException
\end{PRE}

\begin{description}
\htmlDD Sets a parameter for a module. 

\begin{description}
\item {\bf Parameters:}  

modulename - java.lang.String the name of the module.  

parameter - java.lang.String the name of the parameter to set.  

the - value to set for the parameter.  
\end{description}

\end{description}

o {\bf setModuleParameter} 

\begin{PRE}
 public abstract void setModuleParameter(String modulename,
                                         String parameter,
                                         Boolean value) throws CmsException
\end{PRE}

\begin{description}
\htmlDD Sets a parameter for a module. 

\begin{description}
\item {\bf Parameters:}  

modulename - java.lang.String the name of the module.  

parameter - java.lang.String the name of the parameter to set.  

the - value to set for the parameter.  
\end{description}

\end{description}

o {\bf setModuleParameter} 

\begin{PRE}
 public abstract void setModuleParameter(String modulename,
                                         String parameter,
                                         Byte value) throws CmsException
\end{PRE}

\begin{description}
\htmlDD Sets a parameter for a module. 

\begin{description}
\item {\bf Parameters:}  

modulename - java.lang.String the name of the module.  

parameter - java.lang.String the name of the parameter to set.  

the - value to set for the parameter.  
\end{description}

\end{description}

o {\bf setModuleParameter} 

\begin{PRE}
 public abstract void setModuleParameter(String modulename,
                                         String parameter,
                                         Double value) throws CmsException
\end{PRE}

\begin{description}
\htmlDD Sets a parameter for a module. 

\begin{description}
\item {\bf Parameters:}  

modulename - java.lang.String the name of the module.  

parameter - java.lang.String the name of the parameter to set.  

the - value to set for the parameter.  
\end{description}

\end{description}

o {\bf setModuleParameter} 

\begin{PRE}
 public abstract void setModuleParameter(String modulename,
                                         String parameter,
                                         Float value) throws CmsException
\end{PRE}

\begin{description}
\htmlDD Sets a parameter for a module. 

\begin{description}
\item {\bf Parameters:}  

modulename - java.lang.String the name of the module.  

parameter - java.lang.String the name of the parameter to set.  

the - value to set for the parameter.  
\end{description}

\end{description}

o {\bf setModuleParameter} 

\begin{PRE}
 public abstract void setModuleParameter(String modulename,
                                         String parameter,
                                         Integer value) throws CmsException
\end{PRE}

\begin{description}
\htmlDD Sets a parameter for a module. 

\begin{description}
\item {\bf Parameters:}  

modulename - java.lang.String the name of the module.  

parameter - java.lang.String the name of the parameter to set.  

the - value to set for the parameter.  
\end{description}

\end{description}

o {\bf setModuleParameter} 

\begin{PRE}
 public abstract void setModuleParameter(String modulename,
                                         String parameter,
                                         Long value) throws CmsException
\end{PRE}

\begin{description}
\htmlDD Sets a parameter for a module. 

\begin{description}
\item {\bf Parameters:}  

modulename - java.lang.String the name of the module.  

parameter - java.lang.String the name of the parameter to set.  

the - value to set for the parameter.  
\end{description}

\end{description}

o {\bf setModuleParameter} 

\begin{PRE}
 public abstract void setModuleParameter(String modulename,
                                         String parameter,
                                         String value) throws CmsException
\end{PRE}

\begin{description}
\htmlDD Sets a parameter for a module. 

\begin{description}
\item {\bf Parameters:}  

modulename - java.lang.String the name of the module.  

parameter - java.lang.String the name of the parameter to set.  

value - java.lang.String the value to set for the parameter.  
\end{description}

\end{description}

o {\bf setModuleParameter} 

\begin{PRE}
 public abstract void setModuleParameter(String modulename,
                                         String parameter,
                                         boolean value) throws CmsException
\end{PRE}

\begin{description}
\htmlDD Sets a parameter for a module. 

\begin{description}
\item {\bf Parameters:}  

modulename - java.lang.String the name of the module.  

parameter - java.lang.String the name of the parameter to set.  

the - value to set for the parameter.  
\end{description}

\end{description}

o {\bf setModuleParameterdef} 

\begin{PRE}
 public abstract void setModuleParameterdef(String modulename,
                                            Vector names,
                                            Vector descriptions,
                                            Vector types,
                                            Vector values) throws CmsException
\end{PRE}

\begin{description}
\htmlDD Sets the module dependencies for the module. 

\begin{description}
\item {\bf Parameters:}  

module - String the name of the module to check.  

names - Vector with parameternames  

descriptions - Vector with parameterdescriptions  

types - Vector with parametertypes (string, float,...)  

values - Vector with defaultvalues for parameters  
\end{description}

\end{description}

o {\bf setModuleRepositories} 

\begin{PRE}
 public abstract void setModuleRepositories(String modulename,
                                            String repositories[]) throws CmsException
\end{PRE}

\begin{description}
\htmlDD Sets all repositories for a module. 

\begin{description}
\item {\bf Parameters:}  

String - modulname the name of the module.  

String[] - the reprositories of a module.  
\end{description}

\end{description}

o {\bf setModuleVersion} 

\begin{PRE}
 public abstract void setModuleVersion(String modulename,
                                       int version) throws CmsException
\end{PRE}

\begin{description}
\htmlDD This method sets the version of the module. 

\begin{description}
\item {\bf Parameters:}  

String - the name of the module.  

int - the version of the module.  
\end{description}

\end{description}

o {\bf setModuleView} 

\begin{PRE}
 public abstract void setModuleView(String modulename,
                                    String viewname,
                                    String viewurl) throws CmsException
\end{PRE}

\begin{description}
\htmlDD Sets a view for a module 

\begin{description}
\item {\bf Parameters:}  

String - the name of the module.  

String - the name of the view, that is implemented by the module.  

String - the url of the view, that is implemented by the module.  
\end{description}

\end{description}

o {\bf setSystemValue} 

\begin{PRE}
 public abstract void setSystemValue(String dataName,
                                     String value) throws CmsException
\end{PRE}

\begin{description}
\htmlDD Public method to set system values. 

\begin{description}
\item {\bf Parameters:}  

String - dataName the name of the tag to set the data for.  

String - the value to be set.  
\end{description}

\end{description}

o {\bf setSystemValues} 

\begin{PRE}
 public abstract void setSystemValues(String dataName,
                                      Hashtable values) throws CmsException
\end{PRE}

\begin{description}
\htmlDD Public method to set system values with hashtable. 

\begin{description}
\item {\bf Parameters:}  

String - dataName the name of the tag to set the data for.  

Hashtable - the value to be set.  
\end{description}

\end{description}

\htmlHR

\begin{PRE}
All Packages  Class Hierarchy  This Package  Previous  Next  Index
\end{PRE}

