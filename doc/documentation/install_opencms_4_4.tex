\chapter{Installation of OpenCms 4.4}

This chapter provides information on how to install OpenCms using
\rqhttp{http://jakarta.apache.org/tomcat/index.html}{Tomcat} and
\rqhttp{http://www.mysql.com/}{MySql}. All installation parts are
described as single steps. After completing a step you are
strongly advised to verify the success.

\section{Installing Java JDK}
Install Java JDK 1.3.x or later (from SUN
\rqhttp{http://java.sun.com/products/j2se/}{http://java.sun.com/products/j2se/}
or IBM
\rqhttp{http://www3.software.ibm.com}{http://www3.software.ibm.com}).
For details on how to install these components on your operating
system, see the documentation that comes with them.

\section{Installing Tomcat}
Note: OpenCms does not work correctly with Tomcat 3.2.x. Tomcat
3.2.x uses its own XML parser which results in major errors while
using OpenCms. Please use Tomcat 4 instead.

Install JakartaTomcat
\rqhttp{http://jakarta.apache.org/tomcat/index.html}{http://jakarta.apache.org/tomcat/index.html}
into a folder of your choice. This is the JAKARTA\_HOME folder.
Don't forget to set the environment variables JAKARTA\_HOME and
JAVA\_HOME.

Test the installation by running tomcat in standalone mode and
check the examples. Note: Tomcat uses port 8080 in standalone
mode.

If you wish you can combine the servlet-engine with a webserver
like the Apache Web Server
\rqhttp{http://www.apache.org/httpd.html}{http://www.apache.org/httpd.html}.

To make sure Tomcat works with the correct charset to read and
write files you should add the following parameter to the
commandline that is used to start Tomcat:

\texttt{-Dfile.encoding=ISO8859\_1}

You can set the environment varibale CATALINA\_OPTS to do so.
(\texttt{CATALINA\_OPTS=-Dfile.encoding=ISO8859\_1})

\section{Installing MySQL}
Install MySQL from
\rqhttp{http://www.mysql.com/downloads/index.html}{http://www.mysql.com/downloads/index.html}
(see the MySQL documentation on
\rqhttp{http://www.mysql.com/documentation/index.html}{http://www.mysql.com/documentation/index.html}).

Note: On Windows-based systems MySQL has to be installed on the
\texttt{C:\\} drive and should be registered as service using
\texttt{(your MySQL path)/mysql/bin/mysqld -install}.

Start the MySQL server by running the service (WIN32) or executing
\texttt{(your MySQL path)/mysql/bin/mysqld} (UNIX).

Check that MySQL is running before you continue by starting the
MySQL monitor (command \texttt{mysql} in your MySQL bin folder).
The database works correctly if a MySQL prompt appears after
calling the monitor. Quit the MySQL monitor by typing
\texttt{exit} and go to the next step.

\section{Deploy the opencms.war file}
Copy the \texttt{opencms.war} file to
\texttt{CATALINA\_HOME/web-apps/}. Note: replace
\texttt{CATALINA\_HOME} with the real path to your Tomcat
installation.

(Re-)Start Tomcat. Tomcat will now deploy the web application
OpenCms.

\section{Install OpenCms using the Setup-Wizard}
Start the Setup-Wizard by pointing your webbrowser to
\rqhttp{http://localhost:8080/opencms/ocsetup}{http://localhost:8080/opencms/ocsetup}.

Note: Maybe you have to replace \texttt{localhost} with your
servername. The port 8080 is only used if you start Tomcat in
standalone mode.

Follow the instructions of the OpenCms Setup-Wizard. It will set
up the opencms database and import all workplace ressources into
the system. For normal installations with MySql and Tomcat running
on the same server all default settings will fit your needs.

\section{Now your system is ready}
Now your system is ready to use. You can login with user:
\texttt{Admin} password: \texttt{admin}. Please change the
password as soon as possible. The login URL of OpenCms is:
\rqhttp{http://localhost:8080/opencms/opencms/system/workplace/action/login.html}{http://localhost:8080/opencms/opencms/system/workplace/action/login.html}

\section{Security}
After you have installed OpenCms you should look at the security.
Change the Admin password of OpenCms by calling the user
preferences ("hammer" icon on the main screen of the workplace).
Then can add a password to the MySQL database. Enter the following
commands
at the MySQL command line.\\

\begin{quote}
\begin{verbatim}
use mysql; insert into user values ('localhost', 'ocmsuser',
password('XXXXX'),'N','N','N','N','N','N','N','N','N',
'N','N','N','N','N'); insert into db values ('localhost',
'opencms42', 'ocmsuser', 'Y','Y','Y','Y','Y','Y','Y','Y','Y','Y');
flush privileges;
\end{verbatim}
\end{quote}

Don't forget to add the new user and password to all connect
strings of the database to your \texttt{opencms.properties}. The
new user can now only connect to the OpenCms tables. For more
information see the MySQL documentation.

\section{Installation of the example news module}

The setup of the basic system is now complete, but you may want to
add some more functionality to it. A good way to get used to the
system is to practice with the module mechanism and to install a
first module.

A simple newsfeed module came with the basic system or can be
downloaded from the opencms homepage. The installation of the
module is quite easy.

\begin{itemize}
\item Unzip the file \texttt{NewsModule.zip}.
The zip-file contains two files: another zip containinig the real
module (\texttt{com.opencms.modules.homepage.news.zip}) and a
script to set up the MySQL database
(\texttt{NewsModuleSetup.sql}).

\item
First prepare the database.
Change to the folder in which the file \\
\texttt{NewsModuleSetup.sql} resides and type in the following
shell command:

\begin{quote}
\texttt{(your MySQL path)/bin/mysql opencms < NewsModuleSetup.sql}
\end{quote}

This creates all necessary tables and inserts some test data.

\item Then upload the module.
Move the file \\
\texttt{com.opencms.modules.homepage.news.zip} to the
\texttt{modules} subdirectory in your export path.

Open the OpenCms workplace and change to the module management
under the administration view. Make sure that you are within the
online project. Choose the icon to upload a new module. Choose
upload new module from server and continue.

The module should appear in the selectbox. Select
\texttt{continue} to start uploading the module.
\end{itemize}

If all went well, the module is now installed. The module provides
two new administration points. A further documentation can be
found in the virtual filesystem.

Start exploring under
\texttt{/moduledemos/com.opencms.modules.homepage.news/}.


\section{Setting up a local copy of the OpenCms homepage}
You can complete your test system by setting up a local copy of
the OpenCms homepage. The OpenCms homepage uses the news module,
so it is necessary to install the module if you want to make your
first steps with the OpenCms homepage.

To get the templates of the OpenCms homepage you have to make a
simple import. The required import file
(\textit{website\_import.zip}) can be downloaded from the module
page together with another sql-script (\textit{website.sql}). This
script inserts the content which is handled by the News Module
into the database. Both files are bundled to the zip file
\textit{website.zip}.

\begin{itemize}
\item
Extract this file first and then move the file
\textit{website\_import.zip} to your export path.

\item
Then insert the content into the database by using the sql-script.
Change to the folder in which the file \texttt{website.sql}
resides and type in the following shell command:

\begin{quote}
\texttt{(your MySQL path)/bin/mysql opencms < website.sql}
\end{quote}

\item
Then you have to import the templates. To be able to do this, you
have to be within a project (not the online project), so create a
new
project first. \\
Change to the administration view and choose
\texttt{Projectmanagement}
-> \texttt{New Project}.\\
Specify a project name, for example \texttt{OpenCmsHomepage} and
the folders that
should be included. \\
In this case include everything, type in \texttt{/} and press the
button with the blue arrow to adopt this directory to the list of
folders that have to be included.

After creating the new project, make sure that you are actually
inside it and change again to the administration view.

\item
Now start the import of the OpenCms homepage.
Make sure that the import file \\
\texttt{website\_import.zip} can be found within your export
directory. Choose the point \texttt{Databasemanagement} and then
\texttt{import}. Choose the file \texttt{website\_import.zip} from
the selectbox and start the process.

\end{itemize}
That's all. Note: The pictures are missing right now, but they
will be available after publishing the project.
