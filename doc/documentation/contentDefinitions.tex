\chapter{The Idea of Content Definitions}
When writing a template class that produces the output  for a
dynamically created element (e.g. a news element) there is always the
problem where to get the actual content from. For the news example, lets
consider that the news content is taken from a database table.

The straight forward way to access this data would be to implement the
JDBC-Database connection in the template class, but this approach has
several disadvantages:
\begin{itemize}
\item If several template classes have to access the news database,
the DB access code has to be written several times.
\item A change in the DB structure would require changes in several
Java classes.
\item If the news content has to be read form a different kind of
database (or maybe the VFS), the DB access in several template classes
has to be changed.
\end{itemize}

To avoid these problems, the idea is to encapsulate the content and
the access to it in a separate class, the so called Content Definition.

\index{Content Definition}
\index{Back Office}
The Content Definition is an object that contains all fields of the
content itself (in the news example, this would be headline, author,
teaser and the article). This object provides a general interface for manipulating
the data fields of an entry. It has {\meth get()} and {\meth set()} methods to access the data 
in the Content Definition object and a {\meth write()} 
and {\meth delete()} method to handle the data in the data source.
The Content Definition object is used by the template classes to access 
the required data in a general way. 
Therefore, the data source can be modified without changing the template classes at all. 
In addition, the Content Definition is also used to maintain the content
data with an appropriate Backoffice module (see chapter \ref{Backoffice Modules}
about Backoffice modules ).

The way the Content Definition encapsulates the data is very similar to
the way a Java Bean is defined. Because of this, the Content Definition
can be written as a Java Bean as well, enabling its use in other
components of the system.

In a certain way, the CmsObject can be seen as the Content Definition of
the OpenCms itself, it contains methods to access the data of the
system, independent of the means how the data is actually stored.

\section{Writing a simple Content Definition}
Because of the nature of a Content Definition, there are several
different ways how to implement it, especially the access to the data
source always depends on the kind of data source and cannot be
generalized. But the interface that is visible for the programmer using
this Content Definition should be the same for different kind of data and
data sources. This section will describe the purpose and signature 
of the methods to access the data encapsulated in the Content Definition.
This description is the proposal how to implement a Content Definition.
There are other possibilities how to write a Content Definition, but
it is highly recommended to follow the proposals of this document.

A Content Definition object always contains the data of one single entry
of the content type, for the news example already mentioned, this would
be one single news entry, either representing a row in a database table
or a file in the VFS. A news entry could be defined containing the
following content values:

\begin{itemize}
\item ID
\item Title
\item Text
\item Author
\end{itemize}

The simple Content Definition provides five ways to access the
data:
\begin{itemize}
\item Constructors to read a data entry from the data source or to add
a new entry to it.
\item Set and get Methods to access the single data fields.
\item A {\meth write()} method to update an entry in the data source.
\item A {\meth delete()} method to delete an entry in the data source.
\item Static methods that return groups of Content Definition objects.
\end{itemize}

To get a specified news entry, a new Content Definition object has to be
created. This is done by a constructor that takes an argument that can
be used to non-ambiguously select an entry from the data source. In case 
of a database record this is usually the primary key of the table (in 
most cases an integer value). \\
Note: if the Content Definition should be used to manipulate the data in 
the OpenCms Backoffice (from the OpenCms workplace) two constructors have to 
exist.
One that takes the {\class CmsObject} as a parameter and another one
that take the {\class CmsObject} and an Integer object as parameters.
For this reason it is necessary to have a numeric value as a primary key 
in your database table (see section \ref{enhancements} 
for details about adding OpenCms Backoffice functionality).

\begin{java}
\jtaba         /**\\
\jtaba  \xspace * NewsContentDefinition constructor.\\
\jtaba  \xspace * @param id The id of the news entry to be read.\\
\jtaba  \xspace */\\
\jtaba public NewsContentDefinition(CmsObject cms, Integer id) \{\\
\jtabb  // read data from data source and\\
\jtabb  // initialize the member variables here\\
\jtaba \}\\
\end{java}

Creating a new entry is done with a constructor as well. In the next
example, the constructor gets all the data of a news entry and creates 
a new Content Definition object with this data:

\begin{java}
\jtaba         /**\\
\jtaba  \xspace * NewsContentDefinition constructor.\\
\jtaba  \xspace * @param title The title of the news entry.\\
\jtaba  \xspace * @param text The text of the news entry.\\
\jtaba  \xspace * @param author The author of the news entry.\\
\jtaba  \xspace */\\
\jtaba public NewsContentDefinition(String title, String text, String author)\\
\jtaba \{\\
\jtabb        // the object's members are initalized here\\
\jtaba \}\\
\end{java}

This will only create a new Content Definition object, but no entry in
the data source. To add the new entry in the data source, it must be
written by using the {\meth write(}) method of the Content Definition.
Of course the constructor NewsContentDefinition(CmsObject cms) can be
used as well. The data can then be set afterwards by using the
corresponding set-methods. 

On this object, the data stored in the Content Definition can be
accessed by using the different set and get methods, for example the
title field of a news object could be modified or read like this:

\begin{java}
\jtaba          /**\\
\jtaba   \xspace * sets the title of the CD\\
\jtaba   \xspace */\\
\jtaba public void setTitle(String title) \{\\
\jtabb        m\_title=title;\\
\jtaba \}\\[1ex]

\jtaba         /**\\
\jtaba  \xspace * gets the title of the CD\\
\jtaba  \xspace */\\
\jtaba public String getTitle() \{\\
\jtabb        return m\_title;\\
\jtaba \}\\
\end{java}

Set and get methods should be implemented for all data fields of the
Content Definition.

When updating the content of a data field, this data is only updated in
the Content Definition itself. Therefore it is necessary to update the
data source as well. This is done with the {\meth write()} method defined in the
Content Definition which will update all data fields of the current
Content Definition object in the data source.

The {\meth write()} and {\meth delete()} methods to access the data in the
data source have the following signature:
\begin{java}
\jtaba public void write(CmsObject cms);\\[1ex]
\jtaba public void delete(CmsObject cms);\\
\end{java}

Both methods take the {\class CmsObject} as a parameter. This object 
is used to access system-resources in OpenCms and will be used in this
methods to check access rights or get information about the current user
that can also be stored in the data source (for example the information
who has last edited a record).

As stated before, a Content Definition object always stores the data of
a single entry of the content type. It is often required to get a list
of Content Definition objects, e.g. a list of the 10 latest news
entries. This can be achieved by adding static methods to the
Content Definition which return multiple Content Definition objects
(preferable stored in a Vector, Hashtable or an array).  The number
and kind of methods that are implemented this way depends on the
requirements of the stored content and the access to it.

\section{Adding access control to a Content Definition}
A Content Definition can provide a mechanism to control the access to 
the data to prevent unauthorized people from manipulating it. In the news
example every news entry could save access flags to control who has the
rights to read or write a news entry. To use this functionality, information
about the owner, group and access flags for a news entry have to be stored
together with the data for every entry. Therefore three new fields in the data source
and the ContentDefinition are necessary. The OpenCms
provides an abstract class {\class A\_CmsContentDefinition} in the 
{\code com.opencms.defaults} package where some of this functionality is
already implemented. 
This class has the member variable {\name m\_accessFlags} which stores the access rights
for owner group and others in an integer value. The access information can be set with
the method {\meth setAccessFlags(int accessFlags)}. The access right handling uses the
bit-representation of the int-value. For example:
\begin{verbatim} 
access rights: r w v  r w v  r w v  i 
               1 1 1  1 1 1  0 0 0  1 
        value: 1 2 4  8 16     ->   512   = 575
\end{verbatim}
Thus, you have to set the access rights to 575 to get the above settings.
Notice: for certain reasons the bit representation of the integers is just
the other way round as normal!

The class has also some methods to check the access rights for a user.
To check if an entry can be read or write the methods
{\meth isReadable()} and {\meth isWriteable()} are used. This is the default
implementation in the abstract class {\class A\_CmsContentDefinition} of these 
methods:

\begin{java}
\jtaba          /**\\
\jtaba   \xspace * returns true if the CD is readable for the current user\\
\jtaba   \xspace * @returns true\\
\jtaba   \xspace */\\ 
\jtaba public boolean isReadable() \{\\
\jtabb    return true;\\
\jtaba \}\\[1ex]

\jtaba          /**\\
\jtaba   \xspace * returns true if the CD is writeable for the current user\\
\jtaba   \xspace * @returns true\\
\jtaba   \xspace */\\
\jtaba public boolean isWriteable() \{\\
\jtabb    return true;\\
\jtaba \}
\end{java}

As you can see the default implementation of these methods always return true.
So if you want to implement access control in your Content Definition you have
to override these methods and change their default behaviour. An easy way to do
this is to use the already implemented methods {\meth hasReadAccess(CmsObject cms)}
and {\meth hasWriteAccess(CmsObject cms)} of the class {\class A\_CmsContentDefinition}.
These methods implement a standard OpenCms access control by using the already mentioned
fields for user, group and access flags of the ContentDefinition. So you just have to call
these methods in the {\meth isReadable()} and {\meth isWritable()} methods:

\begin{java}
\jtaba      /**\\
\jtaba   \xspace * returns true if the CD is readable for the current user\\
\jtaba   \xspace * @returns true\\
\jtaba   \xspace */\\ 
\jtaba public boolean isReadable() \{\\
\jtabb    return hasReadAccess(m\_cms);\\
\jtaba \}\\[1ex]

\jtaba          /**\\
\jtaba   \xspace * returns true if the CD is writeable for the current user\\
\jtaba   \xspace * @returns true\\
\jtaba   \xspace */\\
\jtaba public boolean isWriteable() \{\\
\jtabb    return hasWriteAcess(m\_cms);\\
\jtaba \}
\end{java}

The variable m\_cms passed to the methods is a member variable of the
Content Definition containing the {\class CmsObject}. This object should
be initialized when creating the Content Definition object with the current
{\class CmsObject}. Because a Content Definition object is usually never reused in 
requests of different users it is save to store the {\class CmsObject} as a member
of the Content Definition. The methods {\meth isReadable()} and {\meth isWritable()}
should be used in the constructors and {\meth write()} and {\meth delete()} methods to check
if the current user has the rights to access the resource.
If you want to implement a different kind of access control
you can do so by calling your own methods inside the {\meth isReadable()}
and {\meth isWritable()} methods. If you want to check the access flags you can use
the constants defined in the {\class I\_CmsConstants} interface. There are constants
like  C\_ACCESS\_OWNER\_READ, C\-\_ACCESS\_GROUP\_READ, C\_ACCESS\_PUBLIC\_READ for the read flag for
user, group and others. The constants for write access are named respectively. Take a look
at the methods {\meth hasReadAccess(CmsObject)} and {\meth hasWriteAccess(CmsObject)}
in the class \\
{\class A\_CmsContentDefinition} to see how this constants can be used.

\section{Content Definition enhancements}
\label{enhancements}
So far, it was shown how to write a simple Content Definition to access
some encapsulated data. In most cases it is necessary to have a Backoffice 
module to maintain the data of this Content Definition. To
simplify the creation of such a Backoffice (see the chapter \ref{Backoffice Modules} about 
Backoffice modules), the Content Definition must be enhanced to offer
several methods that support the implementation of it. To do so, the
Content Definition must extend the abstract class {\class com.opencms.defaults.A\_CmsContentDefinition}
and override some of the methods defined in this class. Here is an excerpt of this
class, showing how these methods are defined:

\begin{java}
public abstract class A\_CmsContentDefinition implements I\_CmsContent\{\\
\jtaba ...\\[1ex]
\jtaba        /**\\
\jtaba \xspace * gets the getXXX methods\\
\jtaba \xspace * @returns a Vector containing the methods\\
\jtaba \xspace */\\
\jtaba public static Vector getFieldMethods() \{\\
\jtabb    return new Vector();\\
\jtaba \}\\[1ex]

\jtaba        /**\\
\jtaba \xspace * gets the headlines of the columns\\
\jtaba \xspace * @returns a Vector with the fields\\
\jtaba \xspace */\\
\jtaba public static Vector getFieldNames() \{\\
\jtabb return new Vector();\\
\jtaba \}\\[1ex]

\jtaba        /**\\
\jtaba \xspace * Gets the filter methods.\\
\jtaba \xspace * You have to override this method in your content definition.\\
\jtaba \xspace * @returns a Vector of FilterMethod objects containing the methods,\\
\jtaba \xspace * names and default parameters\\
\jtaba \xspace */\\
\jtaba public static Vector getFilterMethods() \{\\
\jtabb        return new Vector();\\
\jtaba  \}\\[1ex]

\jtaba        /**\\
\jtaba \xspace * gets a unique id of a content definition object\\
\jtaba \xspace */\\
\jtaba public abstract String getUniqueId();\\
\}
\end{java}

The five methods shown in this code snippet are used to provide some data required by
the Backoffice module for generating the output. Lets consider
the news example: a Backoffice tool would be necessary to add, edit or
delete news entries. Therefore a list of available news entries must be
shown where a single entry can be selected. As there can be a huge
number of entries, this list can be configured with various filters.

Since this kind of display is very common for many different kinds of
content types (news, guestbooks etc.) OpenCms provides some general
classes for generating Backoffice modules. Those classes depend on
information given by Content Definition, like which filters are available
or which labels the columns in the list should have. This information is
provided by the methods that override those of the abstract class:

\begin{itemize}
\index{getFilterMethods()}
\item {\meth getFilterMethods()}: Returns a vector of available filters for
a Content Definition. The data for each filter is stored in an
instance of FilterMethod. These instances carry the data: filtername,
filtermethod and default-parameters for that filter. A filter method
returns a group of Content Definition objects that are a subset of all
available entries, filtered by a specified criteria.

\index{getFieldNames()}
\item {\name getFieldNames()}: This method returns the labels of the columns 
for the list of Content Definition entries generated by the Backoffice classes.

\index{getFieldMethods()}
\item {\meth getFieldMethods()}: The methods of the Content Definition that
return the values of the fields defined in the method above.

\index{getUniqueId()}
\item {\name getUniqueId()}: Returns the unique id of the Content Definition
object. This id is used to identify a single Content Definition object.

\index{isLockable()}
\item {\name isLockable()}: Returns a boolean value indicating if the Content Definition
object should be lockable in the Backoffice. When implementing locking the user-id of the 
user that has locked a resource will be stored in the Content Definition and the data source.
Therefore an additonal member variable and an additional field in the data source are necessary.
Also a method  {\meth getLockstate()} and {\meth setLockstate()} has to be implemented in the 
Content Definition to access the member variable of the Content Definition.
The Backoffice rely upon the implementation of a {\meth getLockstate()} 
and {\meth setLockstate()} method when the {\meth isLockable()} 
method returns true.

\end{itemize}

Despite of these methods a Content Definition that supports the OpenCms
Backoffice functionality has also to provide a constructor that takes the
{\class CmsObject} and an Integer value as parameters. For a class NewsContentDefinition
the constructor would look like this:

\begin{java}
\jtaba        /**\\
\jtaba \xspace * Constructor\\
\jtaba \xspace * @param id the unique id of the entry to read\\
\jtaba \xspace */\\
\jtaba public NewsContentDefinition (CmsObject cms, Integer id) \{\\
\jtabb  // read data from data source and\\
\jtabb  // initialize the member variables here\\
\jtaba \}\\
\end{java}

This constructor is used by the Backoffice classes to create a Content
Definition object when the user edits or deletes an object. So if this
constructor is missing, the Backoffice class will fail to generate a
Content Definition object.\\
The general Backoffice classes are able to generate lists of the data
provided by the Content Definitions.  Since those lists are created
dynamically, the used filters and data fields to be shown must be
defined by the Content Definition in a general way.

The three methods {\meth getFieldNames()}, {\meth getFieldMethods()} and
{\meth getFilterMethods()} are used by the Backoffice classes to generate
the lists of entries. Now we will take a look how an implementation
of these methods can look like. The method 
{\meth getFieldNames()} returns a Vector of Strings that contains the labels 
of the column headings of the created list.
For our news example this method could be implemented this way:

\begin{java}
\jtaba public static Vector getFieldNames() \{\\
\jtabb Vector columns = new Vector();\\
\jtabb columns.addElement("id");\\
\jtabb columns.addElement("title");\\
\jtabb columns.addElement("author");\\
\jtabb return columns;\\
\jtaba \}\\
\end{java}

This would result in a creation of a list with three columns for the id,
title and author in the Backoffice.\\
Note: the Strings stored in the Vector
are the names of the labels in the language-file of the module. So the exact
text that will be shown is defined in a language-file and can easily be 
changed without changing the code in the class. Also you can provide language-files
for multiple languages to have different output when changing to another language in
the workplace (at the moment only german and english is supported by the workplace).
The labels in our language-file could look like this:

\begin{xml}
<?xml version="1.0"?>\\
<LANGUAGE>\\
\xtaba    <mypackage\_NewsBackoffice>\\   
\xtabb      <label>\\
\xtabc        <id>ID</id>\\
\xtabc        <title>Title</title>\\
\xtabc        <author>Author</author>\\
\xtabb      </label>\\
\xtaba    </mypackage\_NewsBackoffice>\\
\xtaba    ...\\
</LANGUAGE>\\   
\end{xml}

The labels have to be defined inside a tag named after the Backoffice class that
creates the lists (fully qualified names with underscores as seperator).
In this example the tag is named \texttt{<mypackage\_NewsBackoffice>}. So the corresponding
class would have the name {\class mypackage.NewsBackoffice} (see chapter \ref{Backoffice Modules} 
about how to implement Backoffice classes).\\
The content of the columns is generated by invoking the methods that are returned
by the method {\meth getFieldMethods()}. In our news example this method could be implemented
this way:

\begin{java}
\jtaba public static Vector getFieldMethods() \{\\
\jtabb        Vector methods = new Vector();\\
\jtabb        Class cd = NewsContentDefinition.class;\\
\jtabb        try \{\\
\jtabc                methods.addElement(cd.getMethod("getTitle", null));\\
\jtabc                methods.addElement(cd.getMethod("getText", null));\\
\jtabc                methods.addElement(cd.getMethod("getAuthor", null));\\
\jtabb        \} catch(NoSuchMethodException exc) \{\\
\jtabc                // ignore the exception\\
\jtabb        \}\\
\jtabb        return methods;\\
\jtaba \}\\
\end{java}

The {\meth getFieldMethods} method returns a Vector of {\class java.lang.reflect.Method} objects.
These objects are used by the Backoffice class to invoke the get methods that return
the values for the columns. The caught {\class NoSuchMethodException} will only 
be thrown if one of the method's names is not typed correctly.\\

As stated before a Content Definition should provide filter-methods that show a 
subset of the available data. For a Content Definition of a news-module you could have for example 
filters that show the entries sorted by the title or a filter that shows
all entries that have an id greater than a given number. The Backoffice class can
generate the list of entries, if it knows about these different filter-methods and
what parameters they have. This information about the filter-methods is given by the
method {\meth getFilterMethods()}. This method returns a Vector of objects of type 
{\class CmsFilterMethod}:\\

\begin{java}
\jtaba public static Vector getFilterMethods(CmsObject cms) \{\\
\jtabb   Vector filter = new Vector();\\
\jtabb   String titleFilter = null;\\
\jtabb   String idFilter = null;\\
\jtabb   try \{\\
\jtabc     CmsXmlLanguageFile lang = new CmsXmlLanguageFile(cms);\\
\jtabd     titleFilter = lang.getLanguageValue(\\
\jtabe       "mypackage\_CmsNewsBackoffice.label.titlefilter");\\
\jtabd     idFilter = lang.getLanguageValue(\\
\jtabe       "mypackage\_CmsNewsBackoffice.label.idfilter");\\
\jtabb   \}catch(Exception e) \{\\
\jtabc    // handle exception that might be thrown \\
\jtabc    // when accessing the language file \\
\jtabb   \}\\
\jtabb   filter.addElement(\\
\jtabc     new CmsFilterMethod(\\
\jtabd      titleFilter,\\
\jtabd       NewsContentDefinition.class.getMethod(\\
\jtabe         "newsByTitle",\\
\jtabe         new Class[] \{CmsObject.class\}\\
\jtabd       ),\\
\jtabd       new Object[0]\\
\jtabc     )\\
\jtabb   );\\
\jtabb   filter.addElement(\\
\jtabc     new CmsFilterMethod(\\
\jtabd       idFilter,\\
\jtabd       NewsContentDefinition.class.getMethod(\\
\jtabe         "newsById",\\
\jtabe         new Class[] \{CmsObject.class, String.class\}\\
\jtabd       ),\\
\jtabd       new Object[0],\\
\jtabd       "1"\\
\jtabc     )\\
\jtabb   );\\
\jtabb   return filter;\\
\jtaba \}
\end{java}

Inside the {\meth getFilterMethods()} two objects of type {\class CmsFilterMethod}
are created and added to a Vector which is then returned. The class
{\class CmsFilterMethod} has two constructors:

\begin{java}
\jtaba public CmsFilterMethod(String filterName, Method filterMethod,\\
\jtabc    Object[] filterParameters)\\[1ex]
\jtaba public CmsFilterMethod(String filterName, Method filterMethod,\\
\jtabc    Object[] filterParameters, String defaultFilterParam)
\end{java}

The parameters of the constructors have the following meanings:\\
\begin{description}
\item[filtername] - the description of the filter as it will appear
in a select-box in the Backoffice. This description can either be defined
in a language-file and then read from there like in the example above, or 
directly coded inside the class. The first approach has the advantage that
it can be changed later without having to change the Java code.
\item[filterMethod] - an instance of {\class java.lang.reflect.Method}
that describes the filtermethod of the Content Definition. You have to give
the name and the type of the parameters as an array of class objects to create
this object.
\item[filterParameters] - this is an array of parameters that will be passed
to the method when it is invoked by the Backoffice class. The parameter of type 
{\class CmsObject} has to be omitted if it is the first parameter of the filtermethod,
it will then be passed automatically to the method. These parameters are useful if you have
a filtermethod that can be used for different filters available in the Backoffice. You can
pass a parameter to control the output and so create multiple filter-entries with one method.
For example you could have an filter that gives back all news entries belonging to a certain
news-category. The filtermethod could have a parameter indicating which category to search for
and by passing a String parameter to the method you could select the category. Normally
you will pass an empty array of objects here if your filtermethod doesn't take a parameter
to "customize" its output.
\item[defaultFilterParam] - the filtermethod can have a user-parameter that is typed 
in an input-field in the Backoffice and then passed to the method. This parameter is a default
value for this parameter, that will be used if the Backoffice list is generated for the first
time when the user couldn't have provided any value for the parameter.\\[1ex]
\textbf{Note:} the Backoffice class checks whether the filtermethod has a user-parameter by comparing
the actual number of parameters of the method as provided by the {\class java.lang.reflect.Method}
object with the number of parameters that should be passed to the method. These are the parameters
given by the parameter \textbf{filterParameters} and additional the {\class CmsObject} 
(if it is the first parameter of the method). If the method has more than these parameters, it is assumed
that a user-parameter is used to provide this parameter.
\end{description}

The filtermethods in the Content Definition for our example would have the following signature:
\begin{java}
\jtaba public Vector newsByTitle (CmsObject cms) \{ .... \}\\[1ex]
\jtaba public Vector newsById (CmsObject cms, String id) \{ ... \}\\
\end{java}






      


